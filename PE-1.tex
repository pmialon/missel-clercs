\subsection{Prière eucharistique I (canon romain)\footnote{la première prière eucharistique est la reprise avec quelques modifications de l'unique 
prière eucharistique du rite romain avant Vatican II, le Canon 
(c'est-à-dire la norme) romain. D'inspiration certainement plus ancienne,
la partie centrale est livrée par saint Ambroise dans les années 380 
et le texte est quasi définitivement fixé avec
saint Grégoire, mort en 604, puis modifié au neuvième siècle.}}\label{pe1} 


\rub{Premiers cierges et Thuriféraire}{s'avancent solennellement devant l'autel}

\rub{tous}{se placent discrètement autour de l'autel}

\subsection{Sanctus\footnote{Par le Sanctus toute la création participe à l'action de grâce            
eucharistique. C'est une bénédiction pour magnifier l'amour trois fois
saint de Dieu.}}

\rep{Sanctus, Sanctus, D\'ominus, Deus S\'abaoth~!\\
Pleni sunt coeli et terra gl\'oria tua. \\
Hos\'anna in excelsis~!\\
Benedictus qui venit in n\'omine D\'omini. \\
Hos\'anna in excelsis~! \\ }

\rep{Saint, Saint, Saint le Seigneur, Dieu de l'univers~!\\
Le ciel et la terre sont remplis de ta gloire.\\
Hosanna au plus haut des cieux~!\\
Béni soit celui qui vient au nom du Seigneur.\\
Hosanna au plus haut des cieux~!  }

\rub{tous}{s'agenouillent en signe d'adoration}


\cel{Après la préface et le Sanctus, le prêtre poursuit, les mains étendues}


Père infiniment bon, toi vers qui montent nos louanges, nous te supplions
par Jésus-Christ, ton Fils, notre Seigneur, \cel{il joint les mains} d'accepter et de bénir 
\cel{Il fait un signe de croix sur le pain et le calice puis étend les mains}
ces offrandes saintes.

Nous te les présentons avant tout pour ta sainte Eglise catholique~:
accorde-lui la paix et protège-la, daigne la rassembler dans l'unité et la
gouverner par toute la terre~; nous les présentons en même temps pour ton
serviteur le Pape N., pour notre Évêque N. et tous ceux
qui veillent fidèlement sur la foi catholique reçue des Apôtres.

Souviens-toi, Seigneur, de tes serviteurs, et de tous 
ceux qui sont ici réunis, dont tu connais la foi et l'attachement. 

\cel{Il joint les mains, prie en silence, puis il reprend, les mains étendues~:}

Nous t'offrons pour eux, ou ils t'offrent pour eux-mêmes et tous les leurs, ce
sacrifice de louange, pour leur propre rédemption, pour le salut qu'ils
espèrent~; et ils te rendent cet hommage, à toi, Dieu éternel, vivant et
vrai.

Dans la communion de toute l'Eglise, nous voulons nommer en premier lieu la
bienheureuse Marie toujours Vierge, Mère de notre Dieu et Seigneur,
Jésus-Christ~; saint Joseph, son époux, les saints Apôtres et Martyrs Pierre
et Paul, André, Jacques et Jean, Thomas, Jacques et Philippe, Barthélemy et
Matthieu, Simon et Jude, Lin, Clet, Clément, Sixte, Corneille et Cyprien,
Laurent, Chrysogone, Jean et Paul, Côme et Damien, et tous les saints.
Accorde-nous, par leur prière et leurs mérites, d'être, toujours et partout,
forts de ton secours et de ta protection.

Dans la communion de toute l'Église,
en ce premier jour de la semaine,
nous célébrons le jour
où le Christ est ressuscité d'entre les morts~;
et nous voulons nommer en premier lieu
la bienheureuse Marie toujours Vierge,
Mère de notre Dieu et Seigneur, Jésus Christ~;

Voici l'offrande que nous présentons devant toi,
nous, tes serviteurs, et ta famille entière~:
dans ta bienveillance, accepte-la.
Assure toi-même la paix de notre vie,
arrache-nous à la damnation
et reçois-nous parmi tes élus.

Sanctifie pleinement cette offrande \cel{Il impose les mains sur les offrandes.} par la puissance de ta bénédiction,
rends-la parfaite et digne de toi~: qu'elle devienne pour nous le corps et
le sang de ton Fils bien-aimé, Jésus-Christ, notre Seigneur.\cel{il joint les mains}

\subsection{Consécration}

\emph{Au cours de la consécration, le pain et le vin deviennent le              
Corps et le Sang du Christ, offerts pour notre salut. Ces paroles sont          
dites à la première personne du singulier car le prêtre agit «~in               
persona Christi~». C'est le Christ lui même en personne qui prononce            
ces paroles en lui.}

La veille de sa passion, il prit le pain dans ses mains très saintes \cel{Il prend le pain.}
et, les yeux levés au ciel,\cel{Il élève les yeux} vers toi, Dieu, son Père tout-puissant, en
te rendant grâce il le bénit, le rompit, et le donna à ses disciples,
en disant~: \textsc{«~Prenez et mangez-en tous~: \cel{Il s'incline un peu} ceci est mon Corps             
livré pour vous.~»}

\cel{Il montre au peuple l'hostie consacrée, la repose sur la patène et fait la génuflexion.}
\rub{Thuriféraire}{encense trois fois trois coups}

De même, à la fin du repas, \cel{Il prend le calice.} il prit dans ses mains cette coupe
incomparable~; et te rendant grâce à nouveau, il la bénit, et la donna
à ses disciples, en disant~: \textsc{«~Prenez et buvez-en tous, \cel{Il s'incline un peu} car             
ceci est la coupe de mon Sang, le sang de l'alliance nouvelle et
éternelle, qui sera versé pour vous et pour la multitude en rémission
des péchés. Vous ferez cela, en mémoire de moi.~»}

\cel{Il montre le calice au peuple, le dépose sur le corporal et fait la génuflexion.}
\rub{Thuriféraire}{encense trois fois trois coups}

\subsection*{Anamnèse\expl{L'assemblée chante le mystère pascal dont la messe est le mémorial.}}

\cel{Puis le prêtre introduit une des trois acclamations suivantes, et le peuple poursuit.}

\begin{itemize}
\item  Il est grand, le mystère de la foi:\\
\rep{Nous proclamons ta mort, Seigneur Jésus,\\
nous célébrons ta résurrection,\\
nous attendons ta venue dans la gloire.}

\item Quand nous mangeons ce pain
et buvons à cette coupe,
nous célébrons le mystère de la foi:\\
\rep{Nous rappelons ta mort,\\
Seigneur ressuscité,\\
et nous attendons que tu viennes.}

\item Proclamons le mystère de la foi:\\
\rep{Gloire à toi qui étais mort,\\
gloire à toi qui es vivant,\\
notre Sauveur et notre Dieu:\\
Viens, Seigneur Jésus!}

\end{itemize}


\cel{Ensuite, les mains étendues, le prêtre dit~:}

C'est pourquoi nous aussi, tes serviteurs, et ton peuple saint avec
nous, faisant mémoire de la passion bienheureuse de ton Fils,
Jésus-Christ, notre Seigneur, de sa résurrection du séjour des morts
et de sa glorieuse ascension dans le ciel, nous te présentons, Dieu de
gloire et de majesté, cette offrande prélevée sur les biens que tu
nous donnes, le sacrifice pur et saint, le sacrifice parfait, pain de
la vie éternelle et coupe du salut.

Et comme il t'a plu d'accueillir les présents d'Abel le Juste, le
sacrifice de notre père Abraham, et celui que t'offrit Melkisédek, ton
grand prêtre, en signe du sacrifice parfait, regarde cette offrande
avec amour et, dans ta bienveillance, accepte-la. 

\cel{Incliné, les mains jointes, il continue~:}

Nous t'en supplions,
Dieu tout-puissant~: qu'elle soit portée par ton ange en présence de
ta gloire, sur ton autel céleste, afin qu'en recevant ici, par notre
communion à l'autel, le corps et le sang de ton Fils, \cel{Il se 
redresse et se signe} nous soyons comblés de ta grâce et de tes 
bénédictions.

\cel{Les mains étendues, il dit~:}

Souviens-toi de tes serviteurs qui nous ont précédés, marqués du signe
de la foi, et qui dorment dans la paix... 

\cel{Il joint les mains et prie en silence, puis il reprend, les mains étendues~:}

Pour eux et pour tous ceux
qui reposent dans le Christ, nous implorons ta bonté~: qu'ils entrent
dans la joie, la paix et la lumière.

Et nous, pécheurs, \cel{Il se frappe la poitrine, puis étend les mains.} 
qui mettons notre espérance en ta miséricorde
inépuisable, admets-nous dans la communauté des bienheureux Apôtres et
Martyrs, de Jean-Baptiste, Etienne, Matthias et Barnabé, Ignace,
Alexandre, Marcellin et Pierre, Félicité et Perpétue, Agathe, Lucie,
Agnès, Cécile, Anastasie, et de tous les saints. Accueille-nous dans
leur compagnie, sans nous juger sur le mérite mais en accordant ton
pardon, par Jésus-Christ, notre Seigneur. \cel{Il joint les mains et continue}

C'est par lui que tu ne cesse de créer tous ces biens, que tu les
bénis, leur donnes la vie, les sanctifies et nous en fais le don.

\subsection{Doxologie}

\emph{Doxologie veut dire parole de gloire. A la fin de la prière               
eucharistique, au nom de l'assemblée, le prêtre va adresser au Père
une parole de gloire.}

\cel{Il prend la patène avec l'hostie, ainsi que le calice, et, les élevant ensemble, il dit:}

Par lui, avec lui et en lui, à toi, Dieu le Père tout-puissant, dans
l'unité du Saint-Esprit, tout honneur et toute gloire, pour les
siècles des siècles.

{\bf Amen.}
