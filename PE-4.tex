\subsection*{Prière Eucharistique IV\footnote{la quatrième prière eucharistique s’inspire de la liturgie orientale
(prière eucharistique de Saint Basile utilisée dans le rite byzantin) et 
reprend, depuis la préface jusqu’à la doxologie finale, toute l’histoire du salut}}\label{pe4}

Vraiment, il est bon de te rendre grâce,
il est juste et bon de te glorifier, Père très saint,
car tu es le seul Dieu, le Dieu vivant et vrai:
tu étais avant tous les siècles,
tu demeures éternellement,
lumière au-delà de toute lumière.
Toi, le Dieu de bonté, la source de la vie,
tu as fait le monde
pour que toute créature
soit comblée de tes bénédictions,
et que beaucoup se réjouissent de ta lumière.
Ainsi, les anges innombrables
qui te servent jour et nuit
se tiennent devant toi,
et, contemplant la splendeur de ta face,
n'interrompent jamais leur louange.
Unis à leur hymne d'allégresse,
avec la création tout entière
qui t'acclame par nos voix,
Dieu, nous te chantons:

\rub{Premiers cierges et Thuriféraire}{s'avancent solennellement devant l'autel}

\rub{tous}{se placent discrètement autour de l'autel}

\subsection{Sanctus\footnote{Par le Sanctus toute la création participe à l'action de grâce            
eucharistique. C'est une bénédiction pour magnifier l'amour trois fois
saint de Dieu.}}

\rep{Sanctus, Sanctus, D\'ominus, Deus S\'abaoth~!\\
Pleni sunt coeli et terra gl\'oria tua. \\
Hos\'anna in excelsis~!\\
Benedictus qui venit in n\'omine D\'omini. \\
Hos\'anna in excelsis~! \\ }

\rep{Saint, Saint, Saint le Seigneur, Dieu de l'univers~!\\
Le ciel et la terre sont remplis de ta gloire.\\
Hosanna au plus haut des cieux~!\\
Béni soit celui qui vient au nom du Seigneur.\\
Hosanna au plus haut des cieux~!  }

\rub{tous}{s'agenouillent en signe d'adoration}

\cel{Le prêtre dit, les mains étendues~:}

Père très saint,
nous proclamons que tu es grand
et que tu as crée toutes choses
avec sagesse et par amour:
tu as fait l'homme à ton image,
et tu lui as confié l'univers.
afin qu'en te servant, toi son Créateur,
il règne sur la création.

Comme il avait perdu ton amitié
en se détournant de toi,
tu ne l'as pas abandonné au pouvoir de la mort.
Dans ta miséricorde,
tu es venu en aide à tous les hommes
pour qu'ils te cherchent et puissent te trouver.
Tu as multiplié les alliances avec eux,
et tu les as formés, par les prophètes,
dans l'espérance du salut.
Tu as tellement aimé le monde,
Père très saint,
que tu nous as envoyé ton propre Fils,
lorsque les temps furent accomplis,
pour qu'il soit notre Sauveur.

Conçu de l'Esprit Saint,
né de la Vierge Marie,
il a vécu notre condition d'homme
en toute chose, excepté le péché,
annonçant aux pauvres
la bonne nouvelle du salut;
aux captifs, la délivrance;
aux affligés, la joie.

Pour accomplir le dessein de ton amour,
il s'est livré lui-même à la mort,
et, par sa résurrection,
il a détruit la mort et renouvelé la vie.
Afin que notre vie ne soit plus à nous-mêmes,
mais à lui qui est mort et ressuscité pour nous,
il a envoyé d'auprès de toi,
comme premier don fait aux croyants,
l'Esprit qui poursuit son œuvre dans le monde
et achève toute sanctification.

\cel{Il rapproche les mains et, en les tenant étendues sur les offrandes, il dit~:}

Que ce même Esprit Saint,
nous t'en prions, Seigneur,
sanctifie ces offrandes:
qu'elles deviennent ainsi \cel{Il fait un signe de croix sur le pain et le calice puis il joint les mains.}
le corps et le sang de ton Fils
dans la célébration de ce grand mystère,
que lui-même nous a laissé
en signe de l'Alliance éternelle.

\subsection*{Consécration}

Quand l'heure fut venue où tu allais le glorifier,
comme il avait aimé les siens qui étaient dans le monde
il les aima jusqu'au bout:
pendant le repas qu'il partageait avec eux,
il prit le pain, Il prend le pain.
il le bénit,
le rompit
et le donna à ses disciples, en disant:

\textsc{<<~Prenez, et mangez-en tous:} \cel{Il s'incline un peu.}
\textsc{ceci est mon corps
livré pour vous.~>>}

\cel{Il montre au peuple l'hostie consacrée, la repose sur la patène et fait la génuflexion.}

\cel{Ensuite il continue:}

De même, il prit la coupe remplie de vin, \cel{Il prend le calice.}
il rendit grâce,
et la donna à ses disciples, en disant: \cel{Il s'incline un peu.}

\textsc{Prenez, et buvez-en tous,
car ceci est la coupe de mon sang,
le sang de l'Alliance nouvelle et éternelle,
qui sera versé
pour vous et pour la multitude
en rémission des péchés.
Vous ferez cela,
en mémoire de moi.}

\cel{Il montre le calice au peuple, le dépose sur le corporal, et fait la génuflexion.}

\subsection*{Anamnèse\expl{L'assemblée chante le mystère pascal dont la messe est le mémorial.}}

\cel{Puis le prêtre introduit une des trois acclamations suivantes, et le peuple poursuit.}

\begin{itemize}
\item  Il est grand, le mystère de la foi:\\
\rep{Nous proclamons ta mort, Seigneur Jésus,\\
nous célébrons ta résurrection,\\
nous attendons ta venue dans la gloire.}

\item Quand nous mangeons ce pain
et buvons à cette coupe,
nous célébrons le mystère de la foi:\\
\rep{Nous rappelons ta mort,\\
Seigneur ressuscité,\\
et nous attendons que tu viennes.}

\item Proclamons le mystère de la foi:\\
\rep{Gloire à toi qui étais mort,\\
gloire à toi qui es vivant,\\
notre Sauveur et notre Dieu:\\
Viens, Seigneur Jésus!}

\end{itemize}


\cel{Ensuite, les mains étendues, le prêtre dit~:}

Voilà pourquoi, Seigneur,
nous célébrons aujourd'hui
le mémorial de notre rédemption:

en rappelant la mort de Jésus Christ
et sa descente au séjour des morts,
en proclamant sa résurection
et son ascention à ta droite dans le ciel,
et attendant aussi
qu''il vienne dans la gloire,
nous t'offrons son corps et son sang,
le sacrifice qui est digne de toi
et qui sauve le monde

Regarde, Seigneur, cette offrande
que tu as donné toi-même à ton Église;
accorde à tous ceux qui vont partager ce pain
et boire à cette coupe
d'être rassemblés par l'Esprit Saint en un seul corps,
pour qu'ils soient eux-même dans le Christ
une vivante offrande
à la louange de ta gloire.

Et maintenant, Seigneur, rapelle-toi
tous ceux pour qui nous offrons le sacrifice:
le Pape N.,
notre évêque N. et tous les évêques,
les prêtres et ceux qui les assistent,
les fidèles qui présentent cette offrande,
les membres de notre assemblée, 
le peuple qui t'appartient
et tous les hommes qui te cherchent avec droiture.

Souviens-toi aussi
de nos frères qui sont morts dans la paix du Christ,
et de tous les morts dont toi seul connais la foi.

A nous qui sommes tes enfants,
accorde, Père très bon,
l'héritage de la vie éternelle
auprès de la Vierge Marie,
la bienheureuse Mère de Dieu,
auprès des Apôtres et de tous les saints,
dans ton Royaume,
où nous pourrons,
avec la création tout entière
enfin libérée du péché et de la mort,
te glorifier
par le Christ, notre Seigneur,
par qui tu donnes au monde
toute grâce et tout bien. \cel{Il joint les mains.}

\subsection{Doxologie}

\emph{Doxologie veut dire parole de gloire. A la fin de la prière               
eucharistique, au nom de l'assemblée, le prêtre va adresser au Père
une parole de gloire.}

\cel{Il prend la patène avec l'hostie, ainsi que le calice, et, les élevant ensemble, il dit:}

Par lui, avec lui et en lui, à toi, Dieu le Père tout-puissant, dans
l'unité du Saint-Esprit, tout honneur et toute gloire, pour les
siècles des siècles.

{\bf Amen.}
