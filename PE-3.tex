\subsection{Prière Eucharistique III\footnote{la troisième prière eucharistique, rédigée en 1967, reprend des 
éléments des traditions gallicanes et hispaniques, sur le plan de la 
prière eucharistique II.}}\label{pe3}

Tu es vraiment saint, Dieu de l'univers,
et toute la création proclame ta louange,
car c'est toi qui donnes la vie,
c'est toi qui sanctifies toutes choses,
par ton Fils, Jésus Christ, notre Seigneur,
avec la puissance de l'Esprit Saint;
et tu ne cesses de rassembler ton peuple,
afin qu'il te présente
partout dans le monde
une offrande pure.

\cel{Il rapproche les mains, et en les tenant étendues sur les offrandes, il dit~:}

C'est pourquoi nous voici rassemblés devant toi,
et, dans la communion de toute l'Église,
en ce premier jour de la semaine
nous célébrons le jour
où le Christ est ressuscité d'entre les morts.
Par lui, que tu as élevé à ta droite,
Dieu tout-puissant, nous te supplions
de consacrer toi-même
les offrandes que nous apportons:

\subsection{Consécration}

La nuit même où il fut livré, il prit le pain, Il prend le pain
en te rendant grâce il le bénit, il le rompit
et le donna à ses disciples, en disant~: 
\textsc{\og~Prenez, et mangez-en tous:} \cel{ Il s'incline un peu.}
\textsc{ceci est mon corps livré pour vous.~\fg}

\cel{Il montre au peuple l'hostie consacrée, la repose sur la patène, et fait la génuflexion. }
\rub{Thuriféraire}{encense trois fois trois coups}


De même, à la fin du repas,
il prit la coupe, \cel{Il prend le calice.}
en te rendant grâce il la bénit,
et la donna à ses disciples, en disant~:
\textsc{\og~Prenez, et buvez-en tous,} \cel{Il s'incline un peu}
\textsc{car ceci est la coupe de mon sang,
le sang de l'Alliance nouvelle et éternelle,
qui sera versé
pour vous et pour la multitude
en rémission des péchés.
Vous ferez cela, en mémoire de moi.~\fg}

\cel{Il montre le calice au peuple, le dépose sur le corporal et fait la génuflexion.}
\rub{Thuriféraire}{encense trois fois trois coups}

\subsection*{Anamnèse\expl{L'assemblée chante le mystère pascal dont la messe est le mémorial.}}

\cel{Puis le prêtre introduit une des trois acclamations suivantes, et le peuple poursuit.}

\begin{itemize}
\item  Il est grand, le mystère de la foi:\\
\rep{Nous proclamons ta mort, Seigneur Jésus,\\
nous célébrons ta résurrection,\\
nous attendons ta venue dans la gloire.}

\item Quand nous mangeons ce pain
et buvons à cette coupe,
nous célébrons le mystère de la foi:\\
\rep{Nous rappelons ta mort,\\
Seigneur ressuscité,\\
et nous attendons que tu viennes.}

\item Proclamons le mystère de la foi:\\
\rep{Gloire à toi qui étais mort,\\
gloire à toi qui es vivant,\\
notre Sauveur et notre Dieu:\\
Viens, Seigneur Jésus!}

\end{itemize}


\cel{Ensuite, les mains étendues, il dit~:}

En faisant mémoire de ton Fils,
de sa passion qui nous sauve,
de sa glorieuse résurrection
et de son ascension dans le ciel,
alors que nous attendons son dernier avènement,
nous présentons cette offrande vivante et sainte
pour te rendre grâce.

Regarde, Seigneur, le sacrifice de ton Église,
et daigne y reconna"tre celui de ton Fils
qui nous a rétablis dans ton Alliance;
quand nous serons nourris de son corps et de son sang
et remplis de l'Esprit Saint,
accorde-nous d'être un seul corps et un seul esprit
dans le Christ.

Que l'Esprit Saint fasse de nous
une éternelle offrande à ta gloire,
pour que nous obtenions un jour
les biens du monde à venir,
auprès de la Vierge Marie,
la bienheureuse Mère de Dieu,
avec les Apôtres, les martyrs, Il peut nommer ici un saint honoré ce jour-là ou le patron du lieu.
(saint N.) et tous les saints,
qui ne cessent d'intercéder pour nous.

Et maintenant, nous te supplions, Seigneur:

Par le sacrifice qui nous réconcilie avec toi,
étends au monde entier le salut et la paix.

Affermis la foi et la charité de ton Église
au long de son chemin sur la terre:
veille sur ton serviteur le Pape N. et notre évêque N.,
l'ensemble des évêques, les prêtres, les diacres,
et tout le peuple des rachetés.

Écoute les prières de ta famille assemblée devant toi,
et ramène à toi, Père très aimant,
tous tes enfants dispersés.

Pour nos frères défunts,
pour les hommes qui ont quitté ce monde
et dont tu connais la droiture, nous te prions:
Reçois-les dans ton Royaume,
où nous espérons être comblés de ta gloire,
tous ensemble et pour l'éternité,
par le Christ, notre Seigneur, Il joint les mains
par qui tu donnes au monde
toute grâce et tout bien.

\subsection{Doxologie}

\emph{Doxologie veut dire parole de gloire. A la fin de la prière               
eucharistique, au nom de l'assemblée, le prêtre va adresser au Père
une parole de gloire.}

\cel{Il prend la patène avec l'hostie, ainsi que le calice, et, les élevant ensemble, il dit:}

Par lui, avec lui et en lui, à toi, Dieu le Père tout-puissant, dans
l'unité du Saint-Esprit, tout honneur et toute gloire, pour les
siècles des siècles.

{\bf Amen.}
