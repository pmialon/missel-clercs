% Type de document, un A4 plié en deux de type article
\documentclass[10pt,twoside,draft,a5paper]{article}
%\documentclass[10pt,twoside,a5paper]{article}

% On définit les types de police pour le document
\usepackage[T1]{fontenc}
\usepackage[utf8]{inputenc}
%\usepackage{amssymb,amsmath}

% On impose les règles de typographie française
\usepackage[francais]{babel}
%\usepackage{frenchle}

% Utilisation d'une extension permettant de faire varier les interlignes
\usepackage{setspace}

% inclusion des graphique en ps
\usepackage{lscape, graphicx, epsfig}

% figure flottante
\usepackage{floatflt}

% utilisation des tableaux à largeur automatique et oui ils existent !
\usepackage{tabularx}

% utilisation du mode multicolonnes
\usepackage{multicol}

% pour les url
\usepackage{url}

% pour avoir la couleur
\usepackage{color}

% Utilisation du souligné  pour pouvoir utiliser \uline
\usepackage[normalem]{ulem}

% package pour les poemes
\usepackage{verse}

% Permettre aux caratères de varier pour s'adapter à la ligne 
\usepackage{microtype}
% \showhyphens{laryngologist dominus}
%\hyphenation{do-mi-nus do-mi-ni}

% Utilisation des header
\usepackage{fancyhdr}
\setcounter{secnumdepth}{-2}
\usepackage[stable]{footmisc}

% On peut aussi définir les marges à la main mais c'est plus bourrin :
% 1in =72pt
\usepackage{layouts}
\usepackage{layout}
\setlength{\voffset}{-36pt}
\setlength{\topmargin}{0pt}
%\setlength{\headheight}{6pt}
%\setlength{\headsep}{6pt}
\setlength{\headheight}{24pt}
\setlength{\headsep}{24pt}

\setlength{\textheight}{\paperheight}
% 54 + 0 + 6 + 6 + 30 + 54 
\addtolength{\textheight}{-150pt}


\setlength{\hoffset}{-18pt}
%\setlength{\oddsidemargin}{-0.3cm}
%\setlength{\evensidemargin}{0.3cm}
\setlength{\textwidth}{\paperwidth}
\addtolength{\textwidth}{-108pt}
\setlength{\marginparsep}{0pt}
\setlength{\marginparwidth}{0pt}


% %%\setlength{\evensidemargin}{-1.0cm}
\setlength{\parindent}{0pt}
% \setlength{\parskip}{0.2cm}
%\setlength{\topmargin}{0cm}
% \setlength{\topmargin}{-2.0cm}



%\setlength{\vindent}{-90pt}

% On définit une nouvelle macro \rub pour les rubriques
\newcommand{\rub}[2]{{\scriptsize \textcolor{red}{\textsc{#1}~: \tiny{#2}}}}
\newcommand{\cel}[1]{{\footnotesize\textcolor{blue}{\emph{#1}}}}
\newcommand{\rep}[1]{{\small\bf
 \begin{verse}[0.9\textwidth]
  #1
 \end{verse}}}
\newcommand{\comment}[1]{\emph{#1}}
\newcommand{\fixme}[1]{{\normalsize\textcolor{green}{\emph{#1}}}}
\newcommand{\siecle}[1]{\begin{math}\textsc{#1}^{e}\end{math}}

% Font Sizes
% 
% \tiny
% \scriptsize
% \footnotesize
% \small
% \normalsize
% \large
% \Large
% \LARGE
% \huge
% \Huge


% On définit le titre dans le header !!!!
\title{Missel à usage des clercs 
(Saint Étienne du mont)
 }

% On définit les auteurs du document 
%\author{Pierre-Gilles \textsc{Mialon} }


\begin{document}
%\setlayoutscale{0.35}
%\pagediagram\currentpage
%\layout

\pagestyle{fancy}
\fancyhead{}
\fancyfoot{}


\fancyhead[LE,RO]{\rightmark}
\fancyhead[LO,RE]{\leftmark}
\fancyfoot[C]{\thepage}

% chant d’entrée (voir Introït), 
% salut du célébrant, 
% acte pénitentiel, 
% Kyrie, 
% Gloria, 
% Collecte, 
% première lecture (Ancien Testament), 
% Psaume ou Graduel, 
% deuxième lec­ture (le plus souvent empruntée à saint Paul), 
% Alleluia, 
% évangile, 
% homélie, 
% Credo, 
% Prière universelle, 
% préparation des dons, 
% Prière sur les offrandes, 
% Prière Eucharistique, 
% Pater et son embolisme, 
% prière et rite de la paix, 
% fraction du pain, 
% communion, 
% Prière après la communion, 
% bénédiction et Renvoi de l’assemblée.

\rub{tous}{preparation de la procession par ordre de taille. Dans l'ordre les premiers cierges puis les deuxième cierge, puis le porte croix, ??? porte missel, thuriféraire, navette, cérémoniaire}
\rub{navette et thuriféraire}{presente l'encens et l'encensoir au célébrant}

\section{Rites initiaux}
\rub{Clerc}{sonne la cloche}

\subsection{Chant d'entrée}

\rub{tous}{procession digne pour se rendre à l'autel.}

\rub{tous}{devant l'autel, salut de celui-ci par une profonde inclination.}

\rub{tous}{rejoignent leur place assise.}

\rub{premier et deuxième cierges avec le porte croix}{ se tiennent debout, derrière l'autel faisant face à celui-ci.}

\rub{thuriféraire}{accompagne le prêtre pour encenser l'autel et la croix.}

\subsection{Signe de Croix.}

\rub{Tous}{Debout, on fait le signe de la croix, tandis que le prêtre dit~:}

Au nom du Père et du Fils et du Saint-Esprit.

\rep{Amen.}

\subsection{Salutation.}

\cel{Le prêtre salue l'assemblée par la formule suivante~:}

La grâce de Jésus notre Seigneur,
l'amour de Dieu le Père
et la communion de l'Esprit Saint
soient toujours avec vous.

\rep{Et avec votre esprit.}


\subsection{Préparation pénitentielle}

\cel{Le prêtre invite les fidèles à la pénitence, en disant~:}

Préparons-nous à la célébration de l'Eucharistie
en reconnaissant que nous sommes pécheurs.

\comment{On fait une brève pause en silence.}

\rep{Je confesse à Dieu tout-puissant,\newline
je reconnais devant mes frères\newline
que j'ai péché\newline
en pensée, en parole,\newline
par action et par omission~;\newline
oui, j'ai vraiment péché.\newline
C'est pourquoi je supplie la Vierge Marie,\newline
les anges et tous les saints,\newline
et vous aussi, mes frères,\newline
de prier pour moi le Seigneur notre Dieu.\newline
}
\cel{Puis le prêtre dit la prière de pardon~:}

Que Dieu tout-puissant
nous fasse miséricorde~;
qu'il nous pardonne nos péchés
et nous conduise à la vie éternelle.

%\rub{tous}{font le signe de croix.}

\rep{Amen.}


\subsection{Kyrie}
\comment{On chante ensuite le Kyrie, prière de supplication. Soit :}

\rep{K\'yrie eléison \\*
Christe eléison \\
K\'yrie eléison  \\}

\rep{Seigneur, prends pitié, \\
Ô Christ, prends pitié, \\
Seigneur, prends pitié. \\
}




\subsection{Gloria\footnote{Ce sont les mots même des anges la nuit de Noël que 
l'on chante à la messe dominicale en dehors des temps de l'Avent et du Carême}}

\rep{Gloria in excelsis Deo\\
Et in terra pax hominibus bonae voluntatis.\\
Laudamus te. Benedicimus te. Adoramus te.\\
Glorificamus te. Gratias agimus tibi\\
propter magnam gloriam tuam,\\
Domine Deus, Rex caelestis,\\
Deus Pater omnipotens.\\
Domine Fili unigenite, Jesu Christe.\\
Domine Deus, Agnus Dei, Filius Patris,\\
qui tollis peccata mundi, miserere nobis.\\
qui tollis peccata mundi, suscipe deprecationem nostram ;\\
qui sedes ad dexteram Patris, miserere nobis.\\
Quoniam tu solus Sanctus,\\
tu solus Dominus,\\
tu solus Altissimus, Jesu Christe.\\
Cum Sancto Spiritu :\\
in gloria Dei Patris. \\
Amen.\\
}

\rep{Gloire à Dieu, au plus haut des cieux, \\
Et paix sur la terre aux hommes qu'il aime.\\
Nous te louons, nous te bénissons, nous t'adorons,\\
Nous te glorifions, nous te rendons grâce,\\
pour ton immense gloire,\\
Seigneur Dieu, Roi du ciel,\\
Dieu le Père tout-puissant.\\
Seigneur, Fils unique, Jésus Christ,\\
Seigneur Dieu, Agneau de Dieu, le Fils du Père.\\
Toi qui enlèves le péché du monde, prends pitié de nous\\
Toi qui enlèves le péché du monde, reçois notre prière ;\\
Toi qui es assis à la droite du Père, prends pitié de nous.\\
Car toi seul es saint,\\
Toi seul es Seigneur,\\
Toi seul es le Très-Haut,\\
Jésus Christ, avec le Saint-Esprit\\
Dans la gloire de Dieu le Père. \\
Amen.\\
}

\subsection{Collecte}

\comment{On l'appelle aussi la prière d’ouverture. Son nom de collecte manifeste son rôle de rassembler la prière de tous. Elle comporte en générale~: \begin{itemize} \item l'invocation louangeuse de Dieu le Père à qui elles s'adressent~:Dieu très bon, Toi qui pardonnes... Père juste, tu nous as aimés...  \item la demande~:  donne à tes enfants de grandir dans l’amour... augmente en nous la foi... accorde-nous le bonheur etc...  \item la doxologie longue où s’affirme la médiation du Christ et la foi trinitaire~: Par Jésus-Christ [...] dans l’Esprit Saint \item l’acquiescement du peuple unanime qui reconnaît dans cette collecte sa propre prière~: Amen \end{itemize}}

\cel{Le prêtre invite les fidèles à la prière~:}

Prions.

... Propre de chaque dimanche

\rep{ Amen.}

\rub{premier et deuxième cierge avec le porte croix}{posent cierges et croix et se dirigent vers leurs places assises.}

\section*{Liturgie de la parole\footnote{L'assemblée se réunit autour des livres qui constituent la                
mémoire vivante de la Parole de Dieu.}}

\emph{On s'assied.}

\subsection*{Première lecture}

...

{\bf Nous rendons grâce à Dieu.}

\subsection*{Psaume}

...

\subsection*{Deuxième lecture}

...


\subsection*{Alléluia}


\emph{Durant le carême on remplace l'Alléluia par un hymne}


\subsection*{Évangile}

...

Acclamons la parole de Dieu.

{\bf Louange à toi, Seigneur Jésus !}


\emph{On s'assied.}


\subsection*{Homélie.}

\emph{Après l'homélie on garde un moment le silence pour méditer ce que l'on vient d'entendre}

\subsection*{Credo}

\subsubsection*{Symbole des Apôtres\footnote{Le Symbole de Apôtres d’une concision bien romaine pourrait remonter au \siecle{ii} siècle. }}

{\bf
\begin{minipage}[t]{0.5\textwidth}
\begin{verse}
Credo in Deum, Patrem omnipotentem, Creatorem caeli et terrae.\\
Et in Iesum Christum, Filium eius unicum, Dominum nostrum:\\
qui conceptus est de Spiritu Sancto, natus ex Maria Virgine,\\
passus sub Pontio Pilato, crucifixus, mortuus, et sepultus,\\
descendit ad inferos,\\
tertia die resurrexit a mortuis,\\
ascendit ad caelos,\\
sedet ad dexteram Dei Patris omnipotentis,\\
inde venturus est iudicare vivos et mortuos.\\
Credo in Spiritum Sanctum,\\
sanctam Ecclesiam catholicam,\\
sanctorum communionem,\\
remissionem peccatorum,\\
carnis resurrectionem,\\
vitam aeternam.\\
Amen.
\end{verse}
\end{minipage}
\begin{minipage}[t]{0.5\textwidth}
\begin{verse}
Je crois en Dieu, le Père tout-puissant, créateur du ciel et de la terre, \\
et en Jésus-Christ, son Fils unique, notre Seigneur, \\
qui a été conçu du Saint-Esprit, (et) qui est né de la Vierge Marie ; \\
(il) a souffert sous Ponce Pilate, (il) a été crucifié, (il) est mort, (il) a été enseveli, (il) est descendu aux enfers ;\\
le troisième jour, (il) est ressuscité des morts ;\\
(il) est monté au ciel, (il) est assis à la droite de Dieu, le Père tout-puissant ; \\
d'où il viendra pour juger les vivants et les morts. \\
Je crois en l'Esprit-Saint \\
à la sainte Église catholique, \\
(à) la communion des saints, \\
(à) la rémission des péchés, \\
(à) la résurrection de la chair \\
et (à) la vie éternelle. \\
Amen.
\end{verse}
\end{minipage}
}




\subsubsection*{Symbole de Nicée\footnote{Le Symbole de Nicée-Constantinople, 
plus long et plus oriental, est l’oeuvre, comme son nom l’indique, des Conciles 
de Nicée (325) et de Constantinople (381)}}

{\bf
\begin{minipage}[t]{0.5\textwidth}
\begin{verse}
 Credo in unum Deum, Patrem omnipoténtem, fact\'orem cæli et terræ, visibílium \'omnium, et invisibílium. \\
 Et in unum D\'ominum Iesum Christum, Fílium Dei unigénitum. \\
 Et ex Patre natum ante \'omnia sæcula. Deum de Deo, lumen de lúmine, Deum verum de Deo vero. \\
 Génitum, non factum, consubstantiálem Patri : per quem \'omnia facta sunt. \\
 Qui propter nos h\'omines, et propter nostram salútem descéndit de cælis. \\
 Et incarnátus est de Spíritu Sancto ex María Vírgine : et homo factus est. \\
 Crucifíxus étiam pro nobis : sub P\'ontio Piláto passus, et sepúltus est. \\
 Et resurréxit tértia die, secúndum Scriptúras. \\
 Et ascéndit in cælum : sedet ad déxteram Patris. \\
 Et íterum ventúrus est cum gl\'oria iudicáre vivos, et m\'ortuos : cuius regni non erit finis. \\
 Et in Spíritum Sanctum, D\'ominum, et vivificántem : qui ex Patre, Fili\'oque procédit. \\
 Qui cum Patre, et Filio simul adorátur, et conglorificátur : qui locútus est per Prophétas. \\
 Et unam, sanctam, cath\'olicam et apost\'olicam Ecclésiam. \\
 Confíteor unum baptísma in remissi\'onem peccat\'orum. \\
 Et expécto resurrecti\'onem mortu\'orum. Et vitam ventúri s\'\ae{}culi.\\
 Amen.
\end{verse}
\end{minipage}
\begin{minipage}[t]{0.5\textwidth}
\begin{verse}
 Je crois en un seul Dieu, le Père tout-puissant, créateur du ciel et de la terre, de l'univers visible et invisible. \\
 Je crois en un seul Seigneur, Jésus-Christ, le Fils unique de Dieu, né du Père avant tous les siècles ;\\
 il est Dieu, né de Dieu, lumière, née de la lumière, vrai Dieu, né du vrai Dieu. \\
 Engendré, non pas créé, de même nature que le Père, et par lui tout a été fait. \\
 Pour nous les hommes, et pour notre salut, il descendit du ciel ; par l'Esprit-Saint, il a pris chair de la Vierge Marie, et s'est fait homme. \\
 Crucifié pour nous sous Ponce Pilate, il souffrit sa passion et fut mis au tombeau. \\
 Il ressuscita le troisième jour, conformément aux Écritures, et il monta au ciel ; il est assis à la droite du Père. \\
 Il reviendra dans la gloire, pour juger les vivants et les morts; et son règne n'aura pas de fin.\\
 Je crois en l'Esprit Saint, qui est Seigneur et qui donne la vie ; il procède du Père et du Fils. \\
 Avec le Père et le Fils, il reçoit même adoration et même gloire ; il a parlé par les prophètes. \\
 Je crois en l'Église, une, sainte, catholique et apostolique. \\
 Je reconnais un seul baptême pour le pardon des péchés. \\
 J'attends la résurrection des morts, et la vie du monde à venir. \\
 Amen.
\end{verse}
\end{minipage}
}



\subsection*{Prière universelle}

...

\emph{Le prêtre conclut cette prière, et tous répondent~:}

{\bf Amen.}




\section{Liturgie eucharistique}

\comment{Eucharistie signifie action de grâce et s'exprime par les                 
quatres verbes du récit de la Cène~: Il prit le pain (préparation des           
dons), il rendit grâce (prière eucharistique), il le rompit (fraction           
du pain) et le donna (communion).}

\subsection{Offertoire}

\rub{Procession des offrandes}{On apporte sur l'autel le pain et le vin qui deviendront le               
Corps et le Sang du Christ.} 
\rub{Cérémoniaire}{déplie le corporal puis se place au côté du célébrant pour disposer les offrandes sur
l'autel}
\rub{S'il n'y a pas de procession des offrandes}{ 
le cérémoniaire distribue les oblats à apporter à la crédence.}
\rub{offrandes}{Le premier clerc à se présenter à la droite du prêtre, sur la marche, 
est toujours celui qui apporte le calice et la patène. Viennent ensuite les coupelles. Puis les burettes.}

\cel{Le prêtre, à l'autel, reçoit la patène avec          
le pain, et il la tient un peu élevée au-dessus de l'autel, en disant           
à voix basse~: }

Tu es béni, Dieu de l'univers, toi qui nous donnes ce pain, fruit de
la terre et du travail des hommes~; nous te le présentons~: il
deviendra le pain de la vie.

\cel{Le prêtre, verse le vin et un peu d'eau dans le calice, en                
disant à voix basse~:}

Comme cette eau se mêle au vin pour le sacrement de l'Alliance,
puissions-nous être unis à la divinité de Celui qui a pris notre
humanité.

\cel{Le prêtre prend le calice, et il le tient un peu élevé au-dessus          
de l'autel, en disant à voix basse~: }

Tu es béni, Dieu de l'univers, toi qui nous donnes ce vin, fruit de la
vigne et du travail des hommes~; nous te le présentons~: il deviendra
le vin du Royaume éternel.

\rub{Cérémoniaire/Thuriféraire et Naviculaire}{sur la marche à droite du célébrant
lui présentent l'encens.}
\rub{Cérémoniaire/Thuriféraire}{accompagne le célébrant qui encense les offrandes}
\rub{Cérémoniaire/Thuriféraire}{encense trois fois deux coups le célébrant, puis les éventuels concélébrants}
\rub{Thuriféraire}{encense la foule trois fois un coup}

\rub{Lavabo}{se dirigent vers le célébrant avec l'eau, la bassine, et le linge}

\rub{porte missel}{dispose le missel sur la gauche de l'autel}

\cel{Incliné, le prêtre dit à voix basse~: }

Humbles et pauvres, nous te supplions, Seigneur, accueille-nous~: que
notre sacrifice, en ce jour, trouve grâce devant toi.

\cel{Puis, sur le côté de l'autel, il se lave les mains, en disant à           
voix basse~:} Lave-moi de mes fautes, Seigneur, purifie-moi de mon
péché.

\subsection{Prière sur les offrandes.}

\comment{On se lève.}

Prions ensemble, au moment d'offrir le sacrifice de toute l'Église.

\rep{Pour la gloire de Dieu et le salut du monde.}

Accueille, Seigneur, le sacrifice que nous t'offrons
pour cette union qu'une loi sainte a consacrée~;
Et puisque tu en es l'auteur,
accepte d'en être aussi le gardien.
Par Jésus, le Christ, notre Seigneur.

\rep{Amen.}


\section{Prière eucharistique.}

\comment{La prière eucharistique est l'action de grâces solennelle au              
cours de laquelle l'Église, par le ministère du prêtre, consacre le             
pain et le vin, les change en corps et sang du Christ, offerts au Père          
pour le salut du monde.}


\subsection{Préface.}
\comment{La préface se compose de quatre parties~: un dialogue entre le            
célébrant et l'assemblée, une action de grâce adressée au Père, une             
action de grâce spécifique à la célébration, puis
l'introduction du sanctus.}

\rub{Premiers cierges et Thuriféraire}{vont se placer pour la procession du Sanctus}

Le Seigneur soit avec vous.

\rep{Et avec votre esprit}

Élevons notre coeur.

\rep{Nous le tournons vers le Seigneur.}

Rendons grâce au Seigneur notre Dieu.

\rep{Cela est juste et bon.}

\subsection{La préface}



\comment{
Vraiment, il est juste et bon de te rendre gloire, voir page \pageref{pe1}\\
Vraiment, Père très saint, il est juste et bon de te rendre grâce, voir page \pageref{pe2} \\
Tu es vraiment saint, Dieu de l'univers, voir page \pageref{pe3} \\
Vraiment, il est bon de te rendre grâce, voir page \pageref{pe4}
}

\subsection{Prière eucharistique I (canon romain)\footnote{la première prière eucharistique est la reprise avec quelques modifications de l'unique 
prière eucharistique du rite romain avant Vatican II, le Canon 
(c'est-à-dire la norme) romain. D'inspiration certainement plus ancienne,
la partie centrale est livrée par saint Ambroise dans les années 380 
et le texte est quasi définitivement fixé avec
saint Grégoire, mort en 604, puis modifié au neuvième siècle.}}\label{pe1} 


\rub{Premiers cierges et Thuriféraire}{s'avancent solennellement devant l'autel}

\rub{tous}{se placent discrètement autour de l'autel}

\subsection{Sanctus\footnote{Par le Sanctus toute la création participe à l'action de grâce            
eucharistique. C'est une bénédiction pour magnifier l'amour trois fois
saint de Dieu.}}

\rep{Sanctus, Sanctus, D\'ominus, Deus S\'abaoth~!\\
Pleni sunt coeli et terra gl\'oria tua. \\
Hos\'anna in excelsis~!\\
Benedictus qui venit in n\'omine D\'omini. \\
Hos\'anna in excelsis~! \\ }

\rep{Saint, Saint, Saint le Seigneur, Dieu de l'univers~!\\
Le ciel et la terre sont remplis de ta gloire.\\
Hosanna au plus haut des cieux~!\\
Béni soit celui qui vient au nom du Seigneur.\\
Hosanna au plus haut des cieux~!  }

\rub{tous}{s'agenouillent en signe d'adoration}


\cel{Après la préface et le Sanctus, le prêtre poursuit, les mains étendues}


Père infiniment bon, toi vers qui montent nos louanges, nous te supplions
par Jésus-Christ, ton Fils, notre Seigneur, \cel{il joint les mains} d'accepter et de bénir 
\cel{Il fait un signe de croix sur le pain et le calice puis étend les mains}
ces offrandes saintes.

Nous te les présentons avant tout pour ta sainte Eglise catholique~:
accorde-lui la paix et protège-la, daigne la rassembler dans l'unité et la
gouverner par toute la terre~; nous les présentons en même temps pour ton
serviteur le Pape N., pour notre Évêque N. et tous ceux
qui veillent fidèlement sur la foi catholique reçue des Apôtres.

Souviens-toi, Seigneur, de tes serviteurs, et de tous 
ceux qui sont ici réunis, dont tu connais la foi et l'attachement. 

\cel{Il joint les mains, prie en silence, puis il reprend, les mains étendues~:}

Nous t'offrons pour eux, ou ils t'offrent pour eux-mêmes et tous les leurs, ce
sacrifice de louange, pour leur propre rédemption, pour le salut qu'ils
espèrent~; et ils te rendent cet hommage, à toi, Dieu éternel, vivant et
vrai.

Dans la communion de toute l'Eglise, nous voulons nommer en premier lieu la
bienheureuse Marie toujours Vierge, Mère de notre Dieu et Seigneur,
Jésus-Christ~; saint Joseph, son époux, les saints Apôtres et Martyrs Pierre
et Paul, André, Jacques et Jean, Thomas, Jacques et Philippe, Barthélemy et
Matthieu, Simon et Jude, Lin, Clet, Clément, Sixte, Corneille et Cyprien,
Laurent, Chrysogone, Jean et Paul, Côme et Damien, et tous les saints.
Accorde-nous, par leur prière et leurs mérites, d'être, toujours et partout,
forts de ton secours et de ta protection.

Dans la communion de toute l'Église,
en ce premier jour de la semaine,
nous célébrons le jour
où le Christ est ressuscité d'entre les morts~;
et nous voulons nommer en premier lieu
la bienheureuse Marie toujours Vierge,
Mère de notre Dieu et Seigneur, Jésus Christ~;

Voici l'offrande que nous présentons devant toi,
nous, tes serviteurs, et ta famille entière~:
dans ta bienveillance, accepte-la.
Assure toi-même la paix de notre vie,
arrache-nous à la damnation
et reçois-nous parmi tes élus.

Sanctifie pleinement cette offrande \cel{Il impose les mains sur les offrandes.} par la puissance de ta bénédiction,
rends-la parfaite et digne de toi~: qu'elle devienne pour nous le corps et
le sang de ton Fils bien-aimé, Jésus-Christ, notre Seigneur.\cel{il joint les mains}

\subsection{Consécration}

\emph{Au cours de la consécration, le pain et le vin deviennent le              
Corps et le Sang du Christ, offerts pour notre salut. Ces paroles sont          
dites à la première personne du singulier car le prêtre agit «~in               
persona Christi~». C'est le Christ lui même en personne qui prononce            
ces paroles en lui.}

La veille de sa passion, il prit le pain dans ses mains très saintes \cel{Il prend le pain.}
et, les yeux levés au ciel,\cel{Il élève les yeux} vers toi, Dieu, son Père tout-puissant, en
te rendant grâce il le bénit, le rompit, et le donna à ses disciples,
en disant~: \textsc{«~Prenez et mangez-en tous~: \cel{Il s'incline un peu} ceci est mon Corps             
livré pour vous.~»}

\cel{Il montre au peuple l'hostie consacrée, la repose sur la patène et fait la génuflexion.}
\rub{Thuriféraire}{encense trois fois trois coups}

De même, à la fin du repas, \cel{Il prend le calice.} il prit dans ses mains cette coupe
incomparable~; et te rendant grâce à nouveau, il la bénit, et la donna
à ses disciples, en disant~: \textsc{«~Prenez et buvez-en tous, \cel{Il s'incline un peu} car             
ceci est la coupe de mon Sang, le sang de l'alliance nouvelle et
éternelle, qui sera versé pour vous et pour la multitude en rémission
des péchés. Vous ferez cela, en mémoire de moi.~»}

\cel{Il montre le calice au peuple, le dépose sur le corporal et fait la génuflexion.}
\rub{Thuriféraire}{encense trois fois trois coups}

\subsection*{Anamnèse\expl{L'assemblée chante le mystère pascal dont la messe est le mémorial.}}

\cel{Puis le prêtre introduit une des trois acclamations suivantes, et le peuple poursuit.}

\begin{itemize}
\item  Il est grand, le mystère de la foi:\\
\rep{Nous proclamons ta mort, Seigneur Jésus,\\
nous célébrons ta résurrection,\\
nous attendons ta venue dans la gloire.}

\item Quand nous mangeons ce pain
et buvons à cette coupe,
nous célébrons le mystère de la foi:\\
\rep{Nous rappelons ta mort,\\
Seigneur ressuscité,\\
et nous attendons que tu viennes.}

\item Proclamons le mystère de la foi:\\
\rep{Gloire à toi qui étais mort,\\
gloire à toi qui es vivant,\\
notre Sauveur et notre Dieu:\\
Viens, Seigneur Jésus!}

\end{itemize}


\cel{Ensuite, les mains étendues, le prêtre dit~:}

C'est pourquoi nous aussi, tes serviteurs, et ton peuple saint avec
nous, faisant mémoire de la passion bienheureuse de ton Fils,
Jésus-Christ, notre Seigneur, de sa résurrection du séjour des morts
et de sa glorieuse ascension dans le ciel, nous te présentons, Dieu de
gloire et de majesté, cette offrande prélevée sur les biens que tu
nous donnes, le sacrifice pur et saint, le sacrifice parfait, pain de
la vie éternelle et coupe du salut.

Et comme il t'a plu d'accueillir les présents d'Abel le Juste, le
sacrifice de notre père Abraham, et celui que t'offrit Melkisédek, ton
grand prêtre, en signe du sacrifice parfait, regarde cette offrande
avec amour et, dans ta bienveillance, accepte-la. 

\cel{Incliné, les mains jointes, il continue~:}

Nous t'en supplions,
Dieu tout-puissant~: qu'elle soit portée par ton ange en présence de
ta gloire, sur ton autel céleste, afin qu'en recevant ici, par notre
communion à l'autel, le corps et le sang de ton Fils, \cel{Il se 
redresse et se signe} nous soyons comblés de ta grâce et de tes 
bénédictions.

\cel{Les mains étendues, il dit~:}

Souviens-toi de tes serviteurs qui nous ont précédés, marqués du signe
de la foi, et qui dorment dans la paix... 

\cel{Il joint les mains et prie en silence, puis il reprend, les mains étendues~:}

Pour eux et pour tous ceux
qui reposent dans le Christ, nous implorons ta bonté~: qu'ils entrent
dans la joie, la paix et la lumière.

Et nous, pécheurs, \cel{Il se frappe la poitrine, puis étend les mains.} 
qui mettons notre espérance en ta miséricorde
inépuisable, admets-nous dans la communauté des bienheureux Apôtres et
Martyrs, de Jean-Baptiste, Etienne, Matthias et Barnabé, Ignace,
Alexandre, Marcellin et Pierre, Félicité et Perpétue, Agathe, Lucie,
Agnès, Cécile, Anastasie, et de tous les saints. Accueille-nous dans
leur compagnie, sans nous juger sur le mérite mais en accordant ton
pardon, par Jésus-Christ, notre Seigneur. \cel{Il joint les mains et continue}

C'est par lui que tu ne cesse de créer tous ces biens, que tu les
bénis, leur donnes la vie, les sanctifies et nous en fais le don.

\subsection{Doxologie}

\emph{Doxologie veut dire parole de gloire. A la fin de la prière               
eucharistique, au nom de l'assemblée, le prêtre va adresser au Père
une parole de gloire.}

\cel{Il prend la patène avec l'hostie, ainsi que le calice, et, les élevant ensemble, il dit:}

Par lui, avec lui et en lui, à toi, Dieu le Père tout-puissant, dans
l'unité du Saint-Esprit, tout honneur et toute gloire, pour les
siècles des siècles.

{\bf Amen.}

% source http://www.clerus.org/clerus/dati/2000-02/14-6/PrexEuch.rtf.html

\subsection{Prière Eucharistique II\footnote{la deuxième prière eucharistique est tirée d’une prière inscrite 
dans un ouvrage du début du troisième siècle : la Tradition apostolique 
(le plus communément attribuée à Hippolyte de Rome vers 2153).}}\label{pe2}

Vraiment, Père très saint,
il est juste et bon de te rendre grâce,
toujours et en tout lieu,
par ton Fils bien-aimé, Jésus Christ:

Car il est ta Parole vivante,
par qui tu as créé toutes choses;
C'est lui que tu nous as envoyé
comme Rédempteur et Sauveur,
Dieu fait homme, conçu de l'Esprit Saint,
né de la Vierge Marie;
Pour accomplir jusqu'au bout ta volonté
et rassembler du milieu des hommes
un peuple saint qui t'appartienne,
il étendit les mains à l'heure de sa passion,
afin que soit brisée la mort,
et que la résurrection soit manifestée.
C'est pourquoi,
avec les anges et tous les saints,
nous proclamons ta gloire,
en chantant (disant) d'une seule voix:

\rub{Premiers cierges et Thuriféraire}{s'avancent solennellement devant l'autel}

\rub{tous}{se placent discrètement autour de l'autel}

\subsection{Sanctus\footnote{Par le Sanctus toute la création participe à l'action de grâce            
eucharistique. C'est une bénédiction pour magnifier l'amour trois fois
saint de Dieu.}}

\rep{Sanctus, Sanctus, D\'ominus, Deus S\'abaoth~!\\
Pleni sunt coeli et terra gl\'oria tua. \\
Hos\'anna in excelsis~!\\
Benedictus qui venit in n\'omine D\'omini. \\
Hos\'anna in excelsis~! \\ }

\rep{Saint, Saint, Saint le Seigneur, Dieu de l'univers~!\\
Le ciel et la terre sont remplis de ta gloire.\\
Hosanna au plus haut des cieux~!\\
Béni soit celui qui vient au nom du Seigneur.\\
Hosanna au plus haut des cieux~!  }

\rub{tous}{s'agenouillent en signe d'adoration}

\cel{Le prêtre dit, les mains étendues~:}

Toi qui es vraiment saint,
toi qui es la source de toute sainteté,
Seigneur, nous te prions:

\cel{Il rapproche les mains et en les tenant étendues sur les offrandes, il dit~:}

Sanctifie ces offrandes
en répandant sur elles ton Esprit; \cel{Il joint les mains}
qu'elles deviennent pour nous \cel{Il fait un signe de croix sur le pain et le calice. Il joint les mains.}
le corps et le sang de Jésus, le Christ, notre Seigneur.

\subsection{Consécration}

Toi qui es vraiment saint,
toi qui es la source de toute sainteté,
nous voici rassemblés devant toi,
et, dans la communion de toute l'Église,
en ce premier jour de la semaine
nous célébrons le jour
où le Christ est ressuscité d'entre les morts.
Par lui que tu as élevé à ta droite,
Dieu notre Père, nous te prions:

Au moment d'être livré
et d'entrer librement dans sa passion,
il prit le pain, il rendit grâce, il le rompit Il prend le pain.
et le donna à ses disciples, en disant~:
\textsc{«~Prenez, et mangez-en tous:} \cel{Il s'incline un peu.}
\textsc{ceci est mon corps livré pour vous.~\fg}

\cel{Il montre au peuple l'hostie consacrée, la repose sur la patène et fait la génuflexion}
\rub{Thuriféraire}{encense trois fois trois coups}



De même, à la fin du repas,
il prit la coupe; \cel{Il prend le calice.}
de nouveau il rendit grâce,
et la donna à ses disciples, en disant~:
\textsc{\og~Prenez, et buvez-en tous,} \cel{Il s'incline un peu.}
\textsc{car ceci est la coupe de mon sang,
le sang de l'Alliance nouvelle et éternelle,
qui sera versé
pour vous et pour la multitude
en rémission des péchés.
Vous ferez cela,
en mémoire de moi.~\fg}

\cel{Il montre le calice au peuple, le dépose sur le corporal et fait la génuflexion.}
\rub{Thuriféraire}{encense trois fois trois coups}

\subsection*{Anamnèse\expl{L'assemblée chante le mystère pascal dont la messe est le mémorial.}}

\cel{Puis le prêtre introduit une des trois acclamations suivantes, et le peuple poursuit.}

\begin{itemize}
\item  Il est grand, le mystère de la foi:\\
\rep{Nous proclamons ta mort, Seigneur Jésus,\\
nous célébrons ta résurrection,\\
nous attendons ta venue dans la gloire.}

\item Quand nous mangeons ce pain
et buvons à cette coupe,
nous célébrons le mystère de la foi:\\
\rep{Nous rappelons ta mort,\\
Seigneur ressuscité,\\
et nous attendons que tu viennes.}

\item Proclamons le mystère de la foi:\\
\rep{Gloire à toi qui étais mort,\\
gloire à toi qui es vivant,\\
notre Sauveur et notre Dieu:\\
Viens, Seigneur Jésus!}

\end{itemize}


\cel{Ensuite, les mains étendues, le prêtre dit:}

Faisant ici mémoire
de la mort et de la résurrection de ton Fils,
nous t'offrons, Seigneur,
le pain de la vie et la coupe du salut,
et nous te rendons grâce,
car tu nous as choisis pour servir en ta présence.

Humblement, nous te demandons
qu'en ayant part au corps et au sang du Christ,
nous soyons rassemblés
par l'Esprit Saint
en un seul corps.

Souviens-toi, Seigneur,
de ton Église répandue à travers le monde:
Fais-la grandir dans ta charité
avec le Pape N.,
notre évêque N.,
et tous ceux qui ont la charge de ton peuple.

Souviens-toi aussi de nos frères
qui se sont endormis
dans l'espérance de la résurrection,
et de tous les hommes qui ont quitté cette vie:
reçois-les dans ta lumière, auprès de toi.

Sur nous tous enfin,
nous implorons ta bonté:
Permets qu'avec la Vierge Marie,
la bienheureuse Mère de Dieu,
avec les Apôtres et les saints de tous les temps
qui ont vécu dans ton amitié,
nous ayons part à la vie éternelle,
et que nous chantions ta louange,
par Jésus Christ, ton Fils bien-aimé. Il joint les mains.

\subsection{Doxologie}

\emph{Doxologie veut dire parole de gloire. A la fin de la prière               
eucharistique, au nom de l'assemblée, le prêtre va adresser au Père
une parole de gloire.}

\cel{Il prend la patène avec l'hostie, ainsi que le calice, et, les élevant ensemble, il dit:}

Par lui, avec lui et en lui, à toi, Dieu le Père tout-puissant, dans
l'unité du Saint-Esprit, tout honneur et toute gloire, pour les
siècles des siècles.

{\bf Amen.}

\subsection{Prière Eucharistique III\footnote{la troisième prière eucharistique, rédigée en 1967, reprend des 
éléments des traditions gallicanes et hispaniques, sur le plan de la 
prière eucharistique II.}}\label{pe3}

Tu es vraiment saint, Dieu de l'univers,
et toute la création proclame ta louange,
car c'est toi qui donnes la vie,
c'est toi qui sanctifies toutes choses,
par ton Fils, Jésus Christ, notre Seigneur,
avec la puissance de l'Esprit Saint;
et tu ne cesses de rassembler ton peuple,
afin qu'il te présente
partout dans le monde
une offrande pure.

\cel{Il rapproche les mains, et en les tenant étendues sur les offrandes, il dit~:}

C'est pourquoi nous voici rassemblés devant toi,
et, dans la communion de toute l'Église,
en ce premier jour de la semaine
nous célébrons le jour
où le Christ est ressuscité d'entre les morts.
Par lui, que tu as élevé à ta droite,
Dieu tout-puissant, nous te supplions
de consacrer toi-même
les offrandes que nous apportons:

\subsection{Consécration}

La nuit même où il fut livré, il prit le pain, Il prend le pain
en te rendant grâce il le bénit, il le rompit
et le donna à ses disciples, en disant~: 
\textsc{\og~Prenez, et mangez-en tous:} \cel{ Il s'incline un peu.}
\textsc{ceci est mon corps livré pour vous.~\fg}

\cel{Il montre au peuple l'hostie consacrée, la repose sur la patène, et fait la génuflexion. Ensuite il continue: }

De même, à la fin du repas,
il prit la coupe, \cel{Il prend le calice.}
en te rendant grâce il la bénit,
et la donna à ses disciples, en disant~:
\textsc{\og~Prenez, et buvez-en tous,} \cel{Il s'incline un peu}
\textsc{car ceci est la coupe de mon sang,
le sang de l'Alliance nouvelle et éternelle,
qui sera versé
pour vous et pour la multitude
en rémission des péchés.
Vous ferez cela, en mémoire de moi.~\fg}

\cel{Il montre le calice au peuple, le dépose sur le corporal et fait la génuflexion.}

\subsection*{Anamnèse\expl{L'assemblée chante le mystère pascal dont la messe est le mémorial.}}

\cel{Puis le prêtre introduit une des trois acclamations suivantes, et le peuple poursuit.}

\begin{itemize}
\item  Il est grand, le mystère de la foi:\\
\rep{Nous proclamons ta mort, Seigneur Jésus,\\
nous célébrons ta résurrection,\\
nous attendons ta venue dans la gloire.}

\item Quand nous mangeons ce pain
et buvons à cette coupe,
nous célébrons le mystère de la foi:\\
\rep{Nous rappelons ta mort,\\
Seigneur ressuscité,\\
et nous attendons que tu viennes.}

\item Proclamons le mystère de la foi:\\
\rep{Gloire à toi qui étais mort,\\
gloire à toi qui es vivant,\\
notre Sauveur et notre Dieu:\\
Viens, Seigneur Jésus!}

\end{itemize}


\cel{Ensuite, les mains étendues, il dit~:}

En faisant mémoire de ton Fils,
de sa passion qui nous sauve,
de sa glorieuse résurrection
et de son ascension dans le ciel,
alors que nous attendons son dernier avènement,
nous présentons cette offrande vivante et sainte
pour te rendre grâce.

Regarde, Seigneur, le sacrifice de ton Église,
et daigne y reconna"tre celui de ton Fils
qui nous a rétablis dans ton Alliance;
quand nous serons nourris de son corps et de son sang
et remplis de l'Esprit Saint,
accorde-nous d'être un seul corps et un seul esprit
dans le Christ.

Que l'Esprit Saint fasse de nous
une éternelle offrande à ta gloire,
pour que nous obtenions un jour
les biens du monde à venir,
auprès de la Vierge Marie,
la bienheureuse Mère de Dieu,
avec les Apôtres, les martyrs, Il peut nommer ici un saint honoré ce jour-là ou le patron du lieu.
(saint N.) et tous les saints,
qui ne cessent d'intercéder pour nous.

Et maintenant, nous te supplions, Seigneur:

Par le sacrifice qui nous réconcilie avec toi,
étends au monde entier le salut et la paix.

Affermis la foi et la charité de ton Église
au long de son chemin sur la terre:
veille sur ton serviteur le Pape N. et notre évêque N.,
l'ensemble des évêques, les prêtres, les diacres,
et tout le peuple des rachetés.

Écoute les prières de ta famille assemblée devant toi,
et ramène à toi, Père très aimant,
tous tes enfants dispersés.

Pour nos frères défunts,
pour les hommes qui ont quitté ce monde
et dont tu connais la droiture, nous te prions:
Reçois-les dans ton Royaume,
où nous espérons être comblés de ta gloire,
tous ensemble et pour l'éternité,
par le Christ, notre Seigneur, Il joint les mains
par qui tu donnes au monde
toute grâce et tout bien.

\subsection{Doxologie}

\emph{Doxologie veut dire parole de gloire. A la fin de la prière               
eucharistique, au nom de l'assemblée, le prêtre va adresser au Père
une parole de gloire.}

\cel{Il prend la patène avec l'hostie, ainsi que le calice, et, les élevant ensemble, il dit:}

Par lui, avec lui et en lui, à toi, Dieu le Père tout-puissant, dans
l'unité du Saint-Esprit, tout honneur et toute gloire, pour les
siècles des siècles.

{\bf Amen.}

\subsection*{Prière Eucharistique IV\expl{la quatrième prière eucharistique s’inspire de la liturgie orientale
(prière eucharistique de Saint Basile utilisée dans le rite byzantin) et 
reprend, depuis la préface jusqu’à la doxologie finale, toute l’histoire du salut}}\label{pe4}

Vraiment, il est bon de te rendre grâce,
il est juste et bon de te glorifier, Père très saint,
car tu es le seul Dieu, le Dieu vivant et vrai:
tu étais avant tous les siècles,
tu demeures éternellement,
lumière au-delà de toute lumière.
Toi, le Dieu de bonté, la source de la vie,
tu as fait le monde
pour que toute créature
soit comblée de tes bénédictions,
et que beaucoup se réjouissent de ta lumière.
Ainsi, les anges innombrables
qui te servent jour et nuit
se tiennent devant toi,
et, contemplant la splendeur de ta face,
n'interrompent jamais leur louange.
Unis à leur hymne d'allégresse,
avec la création tout entière
qui t'acclame par nos voix,
Dieu, nous te chantons:

\rub{Premiers cierges et Thuriféraire}{s'avancent solennellement devant l'autel}

\rub{tous}{se placent discrètement autour de l'autel}

\subsection{Sanctus\footnote{Par le Sanctus toute la création participe à l'action de grâce            
eucharistique. C'est une bénédiction pour magnifier l'amour trois fois
saint de Dieu.}}

\rep{Sanctus, Sanctus, D\'ominus, Deus S\'abaoth~!\\
Pleni sunt coeli et terra gl\'oria tua. \\
Hos\'anna in excelsis~!\\
Benedictus qui venit in n\'omine D\'omini. \\
Hos\'anna in excelsis~! \\ }

\rep{Saint, Saint, Saint le Seigneur, Dieu de l'univers~!\\
Le ciel et la terre sont remplis de ta gloire.\\
Hosanna au plus haut des cieux~!\\
Béni soit celui qui vient au nom du Seigneur.\\
Hosanna au plus haut des cieux~!  }

\rub{tous}{s'agenouillent en signe d'adoration}

\cel{Le prêtre dit, les mains étendues~:}

Père très saint,
nous proclamons que tu es grand
et que tu as crée toutes choses
avec sagesse et par amour:
tu as fait l'homme à ton image,
et tu lui as confié l'univers.
afin qu'en te servant, toi son Créateur,
il règne sur la création.

Comme il avait perdu ton amitié
en se détournant de toi,
tu ne l'as pas abandonné au pouvoir de la mort.
Dans ta miséricorde,
tu es venu en aide à tous les hommes
pour qu'ils te cherchent et puissent te trouver.
Tu as multiplié les alliances avec eux,
et tu les as formés, par les prophètes,
dans l'espérance du salut.
Tu as tellement aimé le monde,
Père très saint,
que tu nous as envoyé ton propre Fils,
lorsque les temps furent accomplis,
pour qu'il soit notre Sauveur.

Conçu de l'Esprit Saint,
né de la Vierge Marie,
il a vécu notre condition d'homme
en toute chose, excepté le péché,
annonçant aux pauvres
la bonne nouvelle du salut;
aux captifs, la délivrance;
aux affligés, la joie.

Pour accomplir le dessein de ton amour,
il s'est livré lui-même à la mort,
et, par sa résurrection,
il a détruit la mort et renouvelé la vie.
Afin que notre vie ne soit plus à nous-mêmes,
mais à lui qui est mort et ressuscité pour nous,
il a envoyé d'auprès de toi,
comme premier don fait aux croyants,
l'Esprit qui poursuit son œuvre dans le monde
et achève toute sanctification.

\cel{Il rapproche les mains et, en les tenant étendues sur les offrandes, il dit~:}

Que ce même Esprit Saint,
nous t'en prions, Seigneur,
sanctifie ces offrandes:
qu'elles deviennent ainsi \cel{Il fait un signe de croix sur le pain et le calice puis il joint les mains.}
le corps et le sang de ton Fils
dans la célébration de ce grand mystère,
que lui-même nous a laissé
en signe de l'Alliance éternelle.

\subsection*{Consécration}

Quand l'heure fut venue où tu allais le glorifier,
comme il avait aimé les siens qui étaient dans le monde
il les aima jusqu'au bout:
pendant le repas qu'il partageait avec eux,
il prit le pain, Il prend le pain.
il le bénit,
le rompit
et le donna à ses disciples, en disant~: 
\textsc{\og~Prenez, et mangez-en tous:} \cel{Il s'incline un peu.}
\textsc{ceci est mon corps livré pour vous.~\fg} 

\cel{Il montre au peuple l'hostie consacrée, la repose sur la patène et fait la génuflexion.}

\rub{Thuriféraire}{encense trois fois trois coups}

\cel{Ensuite il continue:}

De même, il prit la coupe remplie de vin, \cel{Il prend le calice.}
il rendit grâce,
et la donna à ses disciples, en disant~: \textsc{\og~Prenez, et buvez-en tous,} \cel{Il s'incline un peu.}
\textsc{car ceci est la coupe de mon sang,
le sang de l'Alliance nouvelle et éternelle,
qui sera versé
pour vous et pour la multitude
en rémission des péchés.
Vous ferez cela,
en mémoire de moi.~\fg}

\cel{Il montre le calice au peuple, le dépose sur le corporal, et fait la génuflexion.}

\rub{Thuriféraire}{encense trois fois trois coups}

\subsection*{Anamnèse\expl{L'assemblée chante le mystère pascal dont la messe est le mémorial.}}

\cel{Puis le prêtre introduit une des trois acclamations suivantes, et le peuple poursuit.}

\begin{itemize}
\item  Il est grand, le mystère de la foi:\\
\rep{Nous proclamons ta mort, Seigneur Jésus,\\
nous célébrons ta résurrection,\\
nous attendons ta venue dans la gloire.}

\item Quand nous mangeons ce pain
et buvons à cette coupe,
nous célébrons le mystère de la foi:\\
\rep{Nous rappelons ta mort,\\
Seigneur ressuscité,\\
et nous attendons que tu viennes.}

\item Proclamons le mystère de la foi:\\
\rep{Gloire à toi qui étais mort,\\
gloire à toi qui es vivant,\\
notre Sauveur et notre Dieu:\\
Viens, Seigneur Jésus!}

\end{itemize}


\cel{Ensuite, les mains étendues, le prêtre dit~:}

Voilà pourquoi, Seigneur,
nous célébrons aujourd'hui
le mémorial de notre rédemption~:

en rappelant la mort de Jésus Christ
et sa descente au séjour des morts,
en proclamant sa résurection
et son ascention à ta droite dans le ciel,
et attendant aussi
qu'il vienne dans la gloire,
nous t'offrons son corps et son sang,
le sacrifice qui est digne de toi
et qui sauve le monde

Regarde, Seigneur, cette offrande
que tu as donné toi-même à ton Église;
accorde à tous ceux qui vont partager ce pain
et boire à cette coupe
d'être rassemblés par l'Esprit Saint en un seul corps,
pour qu'ils soient eux-même dans le Christ
une vivante offrande
à la louange de ta gloire.

Et maintenant, Seigneur, rapelle-toi
tous ceux pour qui nous offrons le sacrifice:
le Pape N.,
notre évêque N. et tous les évêques,
les prêtres et ceux qui les assistent,
les fidèles qui présentent cette offrande,
les membres de notre assemblée, 
le peuple qui t'appartient
et tous les hommes qui te cherchent avec droiture.

Souviens-toi aussi
de nos frères qui sont morts dans la paix du Christ,
et de tous les morts dont toi seul connais la foi.

A nous qui sommes tes enfants,
accorde, Père très bon,
l'héritage de la vie éternelle
auprès de la Vierge Marie,
la bienheureuse Mère de Dieu,
auprès des Apôtres et de tous les saints,
dans ton Royaume,
où nous pourrons,
avec la création tout entière
enfin libérée du péché et de la mort,
te glorifier
par le Christ, notre Seigneur,
par qui tu donnes au monde
toute grâce et tout bien. \cel{Il joint les mains.}

\subsection{Doxologie}

\emph{Doxologie veut dire parole de gloire. A la fin de la prière               
eucharistique, au nom de l'assemblée, le prêtre va adresser au Père
une parole de gloire.}

\cel{Il prend la patène avec l'hostie, ainsi que le calice, et, les élevant ensemble, il dit:}

Par lui, avec lui et en lui, à toi, Dieu le Père tout-puissant, dans
l'unité du Saint-Esprit, tout honneur et toute gloire, pour les
siècles des siècles.

{\bf Amen.}



\section{La communion}

\subsection{Notre Père.}

\cel{Le prêtre introduit la prière du Seigneur en disant, par                  
exemple~: }

\rub{tous}{se disposent harmonieusement derrière l'autel sous le regard du Cérémoniaire}

Comme nous l'avons appris du Sauveur, et selon son commandement, nous
osons dire~:

\rep{Notre Père qui es aux cieux,\\
que ton nom soit sanctifié,\\
que ton règne vienne,\\
que ta volonté soit faite\\
sur la terre comme ciel.\\
Donne-nous aujourd'hui\\
notre pain de ce jour.\\
Pardonne-nous nos offenses,\\
comme nous pardonnons aussi\\
à ceux qui nous ont offensés.\\
Et ne nous soumets pas à la tentation,\\
mais délivre nous du Mal.\\
}

\subsection{Rite de la paix}

Seigneur Jésus-Christ, tu as dit à tes Apôtres~: «~Je vous laisse la            
paix, je vous donne ma paix~»~; ne regarde pas nos péchés, mais la
foi de ton Église~; pour que ta volonté s'accomplisse, donne-lui
toujours cette paix, et conduis-la vers l'unité parfaite, toi qui
règnes pour les siècles des siècles.

\rep{ Amen.}

Que la paix du Seigneur soit toujours avec vous.

\rep{ Et avec votre esprit.}

Dans la charité du Christ, donnez-vous la paix.

\fixme{statuer sur ce que l'on fait}

\subsection{Fraction du pain}

\emph{Le prêtre rompt le pain consacré et en met un fragment dans le            
calice, endisant à voix basse~:}

Que le corps et le sang de Jésus-Christ, réunis dans cette coupe,
nourrissent en nous la vie éternelle.

\subsection{Agnus}

\rep{Agnus Dei qui tollis peccata mundi, miserere nobis \\
Agnus Dei qui tollis peccata mundi, miserere nobis \\
Agnus Dei qui tollis peccata mundi, dona nobis pacem \\
}

\rep{Agneau de Dieu, qui enlèves le péché du monde, prends pitié de nous,\\
Agneau de Dieu, qui enlèves le péché du monde, prends pitié de nous,\\
Agneau de Dieu, qui enlèves le péché du monde, donne-nous la paix.\\
}

\rub{tous}{se disposent en procession à une distance confortable derrière l'autel et s'agenouillent}


\subsection{Communion}

\cel{Le prêtre dit à voix basse une prière de préparation à la                 
communion~:}

Seigneur Jésus Christ, Fils du Dieu vivant, selon la volonté du Père
et avec la puissance du Saint-Esprit, tu as donné, par ta mort, la vie
au monde~; que ton corps et ton sang me délivrent de mes péchés et de
tout mal~; fais que je demeure fidèle à tes commandements et que
jamais je ne sois séparé de toi.
Heureux les invités au festin des noces de l'Agneau !
Voici l'Agneau de Dieu qui enlève le péché du monde.

\rep{ Seigneur, je ne suis pas digne de te recevoir~;                            
mais dis seulement une parole et je serai guéri.}

\cel{ Puis le prêtre dit à voix basse~:}
Que le corps du Christ me garde pour la vie éternelle.

\cel{ Ensuite, il prend le calice, et dit à voix basse~:}
Que le sang du Christ me garde pour la vie éternelle.

\rub{Cérémoniaire}{invite le célébrant à donner la communion aux clercs}
\rub{tous}{se relèvent et communient, ils se disposent ensuite au point de communion}

\rub{Cérémoniaire}{reste à sa place}

\comment{Les fidèles qui le veulent s'avancent pour communier.}

\subsection{Chant de communion}

\comment{Lorsque la distribution de la communion est achevée}

\rub{Cérémoniaire assisté de deux servants}{se présentent à l'autel ou à la crédence avec la burette d'eau}

\cel{Le prêtre purifie la patène et le calice, en disant à voix basse~:}

Puissions-nous accueillir d'un coeur pur, Seigneur, ce que notre
bouche a reçu, et trouver dans cette communion d'ici-bas la guérison
pour la vie éternelle.

\rub{tous}{accompagnent du regard le retour du Saint Sacrement au Tabernacle. 
Ils genuflexent en même temps que le prêtre} 

\subsection{Prière après la communion}

\cel{On se lève. Le prêtre invite les fidèles à la prière~:}

Prions.

...

\rep{ Amen.}

\section{Rite de conclusion}

\cel{Le prêtre bénit les fidèles.}

Le Seigneur soit avec vous.

\rep{ Et avec votre esprit.}

...

\rep{ Amen.}



\subsection{Envoi}

\cel{Le prêtre renvoie l'assemblée~:}

Allez dans la paix du Christ.

\rep{ Nous rendons grâce à Dieu.}

\subsection{Sortie}



\end{document}
