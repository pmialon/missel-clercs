\rub{tous}{Préparation de la procession par ordre de taille. 
Dans l'ordre les premiers cierges, le porte croix, les deuxièmes cierges,
les acolytes, enfin thuriféraire et naviculaire}

\rub{cérémoniaire}{précède immédiatement le clergé, s'il n'a pas d'autres fonctions.}

\rub{navette et thuriféraire}{Lorsque le prêtre sort de la sacristie, ils présentent 
l'encens et l'encensoir au célébrant}

\section*{Rites initiaux}
\rub{Un Clerc}{sonne la cloche trois coups espacés}

\subsection*{Chant d'entrée\expl{Pour accueillir le Christ qui vient}}

\rub{tous}{procession digne pour se rendre à l'autel.}

\rub{tous}{devant l'autel, salut de celui-ci par une profonde inclination. }

\rub{premier et deuxième cierges avec le porte croix}{saluent l'autel par une légère inclination de la tête 
puis se tiennent debout, derrière l'autel faisant face à celui-ci.}

\rub{cérémoniaire}{salue l'autel en même temps que le célébrant.}

\rub{les autres}{rejoignent leur place assise.}

\rub{cérémoniaire/thuriféraire}{remet l'encensoir au prêtre se tenant sur sa droite et l'accompagne 
pour encenser l'autel et la croix.}

\subsection*{Signe de Croix.}

\rub{Tous}{Debout, on fait le signe de la croix, tandis que le prêtre dit~:}

Au nom du Père et du Fils et du Saint-Esprit.

\rep{Amen.}

\subsection*{Salutation.}

\cel{Le prêtre salue l'assemblée par la formule suivante~:}

La grâce de Jésus notre Seigneur,
l'amour de Dieu le Père
et la communion de l'Esprit Saint
soient toujours avec vous.

\rep{Et avec votre esprit.}


\subsection*{Préparation pénitentielle}

\cel{Le prêtre invite les fidèles à la pénitence, en disant~:}

Préparons-nous à la célébration de l'Eucharistie
en reconnaissant que nous sommes pécheurs.

\comment{On fait une brève pause en silence.}

\rep{Je confesse à Dieu tout-puissant,\\
je reconnais devant mes frères\\
que j'ai péché\\
en pensée, en parole,\\
par action et par omission~;\\
oui, j'ai vraiment péché.\\
\rub{tous}{se frappent la poitrine avec la main droite} \\
C'est pourquoi je supplie la Vierge Marie,\\
les anges et tous les saints,\\
et vous aussi, mes frères,\\
de prier pour moi le Seigneur notre Dieu.\\
}
\cel{Puis le prêtre dit la prière de pardon~:}

Que Dieu tout-puissant
nous fasse miséricorde~;
qu'il nous pardonne nos péchés
et nous conduise à la vie éternelle.

%\rub{tous}{font le signe de croix.}

\rep{Amen.}


\subsection*{Kyrie}
\comment{On chante ensuite le Kyrie, prière de supplication. Soit :}

\rep{K\'yrie eléison \\*
Christe eléison \\
K\'yrie eléison  \\}

\rep{Seigneur, prends pitié, \\
Ô Christ, prends pitié, \\
Seigneur, prends pitié. \\
}




\subsection*{Gloria\expl{Ce sont les mots même des anges la nuit de Noël que 
l'on chante à la messe dominicale en dehors des temps de l'Avent et du Carême}}

\rep{Gloria in excelsis Deo\\
Et in terra pax hominibus bonae voluntatis.\\
Laudamus te. Benedicimus te. Adoramus te.\\
Glorificamus te. Gratias agimus tibi\\
propter magnam gloriam tuam,\\
Domine Deus, Rex caelestis,\\
Deus Pater omnipotens.\\
Domine Fili unigenite, Jesu Christe.\\
\rub{tous}{inclinent la tête}\\
Domine Deus, Agnus Dei, Filius Patris,\\
qui tollis peccata mundi, miserere nobis.\\
qui tollis peccata mundi, suscipe deprecationem nostram ;\\
qui sedes ad dexteram Patris, miserere nobis.\\
Quoniam tu solus Sanctus,\\
tu solus Dominus,\\
tu solus Altissimus, Jesu Christe.\\
\rub{tous}{inclinent la tête}\\
\rub{porte missel}{apporte le missel à la page ouverte au célébrant}\\
Cum Sancto Spiritu :\\
in gloria Dei Patris. \\
Amen.\\
}

\rep{Gloire à Dieu, au plus haut des cieux, \\
Et paix sur la terre aux hommes qu'il aime.\\
Nous te louons, nous te bénissons, nous t'adorons,\\
Nous te glorifions, nous te rendons grâce,\\
pour ton immense gloire,\\
Seigneur Dieu, Roi du ciel,\\
Dieu le Père tout-puissant.\\
Seigneur, Fils unique, Jésus Christ,\\
Seigneur Dieu, Agneau de Dieu, le Fils du Père.\\
Toi qui enlèves le péché du monde, prends pitié de nous\\
Toi qui enlèves le péché du monde, reçois notre prière~;\\
Toi qui es assis à la droite du Père, prends pitié de nous.\\
Car toi seul es saint,\\
Toi seul es Seigneur,\\
Toi seul es le Très-Haut,\\
\rub{porte missel}{apporte le missel à la page ouverte au célébrant}\\
Jésus Christ, avec le Saint-Esprit\\
Dans la gloire de Dieu le Père. \\
Amen.\\
}



\subsection*{Collecte\expl{On l'appelle aussi la prière d’ouverture. Son nom de collecte manifeste son rôle de rassembler la prière de tous.}}

\cel{Le prêtre invite les fidèles à la prière~:}

Prions.

\comment{... Propre de chaque dimanche}

\rub{porte missel}{incline la tête pendant la doxologie}

\rep{ Amen.}

\rub{premiers et deuxièmes cierges avec le porte croix}{posent cierges et croix et se dirigent vers leurs 
places assises.}
