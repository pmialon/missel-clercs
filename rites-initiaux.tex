\section*{Ouverture de la célébration}
\rub{Clerc}{sonne la cloche}

\subsection*{Chant d'entrée}

\rub{tous}{procession digne pour se rendre à l'autel}

\subsection*{Signe de Croix.}


\rub{Tous}{Debout, on fait le signe de la croix, tandis que le prêtre dit~:}

Au nom du Père et du Fils et du Saint-Esprit.

{\bf Amen.}

\subsection*{Salutation.}

\emph{Le prêtre salue l'assemblée par la formule suivante~:}

La grâce de Jésus notre Seigneur,
l'amour de Dieu le Père
et la communion de l'Esprit Saint
soient toujours avec vous.

{\bf Et avec votre esprit.}


\subsection*{Préparation pénitentielle}

\emph{Le prêtre invite les fidèles à la pénitence, en disant~:}

Préparons-nous à la célébration de l'Eucharistie
en reconnaissant que nous sommes pécheurs.

\emph{On fait une brève pause en silence.}

\begin{verse}
\noindent
Je confesse à Dieu tout-puissant,\newline
je reconnais devant mes frères\newline
que j'ai péché\newline
en pensée, en parole,\newline
par action et par omission~;\newline
oui, j'ai vraiment péché.\newline
C'est pourquoi je supplie la Vierge Marie,\newline
les anges et tous les saints,\newline
et vous aussi, mes frères,\newline
de prier pour moi le Seigneur notre Dieu.\newline
\end{verse}

\emph{Puis le prêtre dit la prière de pardon~:}

Que Dieu tout-puissant
nous fasse miséricorde~;
qu'il nous pardonne nos péchés
et nous conduise à la vie éternelle.

{\bf Amen.}



\subsection*{Kyrie}
\emph{On chante ensuite le Kyrie, prière de supplication. Soit :}





\subsection*{Gloria\footnote{Ce sont les mots même des anges la nuit de Noël que 
l'on chante à la messe dominicale en dehors des temps de l'Avent et du Carême}}

\begin{verse}
Gloria in excelsis Deo\\
Et in terra pax hominibus bonae voluntatis.\\
Laudamus te. Benedicimus te. Adoramus te.\\
Glorificamus te. Gratias agimus tibi\\
propter magnam gloriam tuam,\\
Domine Deus, Rex caelestis,\\
Deus Pater omnipotens.\\
Domine Fili unigenite, Jesu Christe.\\
Domine Deus, Agnus Dei, Filius Patris,\\
qui tollis peccata mundi, miserere nobis.\\
qui tollis peccata mundi, suscipe deprecationem nostram ;\\
qui sedes ad dexteram Patris, miserere nobis.\\
Quoniam tu solus Sanctus,\\
tu solus Dominus,\\
tu solus Altissimus, Jesu Christe.\\
Cum Sancto Spiritu :\\
in gloria Dei Patris. \\
Amen.\\
\end{verse}

\emph{Traduction du latin}
\begin{verse}
Gloire à Dieu, au plus haut des cieux, \\
Et paix sur la terre aux hommes qu'il aime.\\
Nous te louons, nous te bénissons, nous t'adorons,\\
Nous te glorifions, nous te rendons grâce,\\
pour ton immense gloire,\\
Seigneur Dieu, Roi du ciel,\\
Dieu le Père tout-puissant.\\
Seigneur, Fils unique, Jésus Christ,\\
Seigneur Dieu, Agneau de Dieu, le Fils du Père.\\
Toi qui enlèves le péché du monde, prends pitié de nous\\
Toi qui enlèves le péché du monde, reçois notre prière ;\\
Toi qui es assis à la droite du Père, prends pitié de nous.\\
Car toi seul es saint,\\
Toi seul es Seigneur,\\
Toi seul es le Très-Haut,\\
Jésus Christ, avec le Saint-Esprit\\
Dans la gloire de Dieu le Père. \\
Amen.\\
\end{verse}



\subsection*{Collecte
  \footnote{On l'appelle aussi la prière d’ouverture. 
    Son nom de collecte manifeste son rôle de rassembler la prière de tous. 
    Elle comporte en générale : 
    \begin{itemize}
      \item l'invocation louangeuse de Dieu le Père à qui elles s'adressent~:Dieu très bon, Toi qui pardonnes... Père juste, tu nous as aimés... 
      \item la demande~:  donne à tes enfants de grandir dans l’amour... augmente en nous la foi... accorde-nous le bonheur etc... 
      \item la doxologie longue où s’affirme la médiation du Christ et la foi trinitaire~: Par Jésus-Christ [...] dans l’Esprit Saint 
      \item l’acquiescement du peuple unanime qui reconnaît dans cette collecte sa propre prière~: Amen 
    \end{itemize}
  }}

\emph{Le prêtre invite les fidèles à la prière~:}

Prions.

... Propre de chaque dimanche

{\bf Amen.}


