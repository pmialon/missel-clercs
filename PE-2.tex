% source http://www.clerus.org/clerus/dati/2000-02/14-6/PrexEuch.rtf.html

\subsection{Prière Eucharistique II\footnote{la deuxième prière eucharistique est tirée d’une prière inscrite 
dans un ouvrage du début du troisième siècle : la Tradition apostolique 
(le plus communément attribuée à Hippolyte de Rome vers 2153).}}\label{pe2}

Vraiment, Père très saint,
il est juste et bon de te rendre grâce,
toujours et en tout lieu,
par ton Fils bien-aimé, Jésus Christ:

Car il est ta Parole vivante,
par qui tu as créé toutes choses;
C'est lui que tu nous as envoyé
comme Rédempteur et Sauveur,
Dieu fait homme, conçu de l'Esprit Saint,
né de la Vierge Marie;
Pour accomplir jusqu'au bout ta volonté
et rassembler du milieu des hommes
un peuple saint qui t'appartienne,
il étendit les mains à l'heure de sa passion,
afin que soit brisée la mort,
et que la résurrection soit manifestée.
C'est pourquoi,
avec les anges et tous les saints,
nous proclamons ta gloire,
en chantant (disant) d'une seule voix:

\rub{Premiers cierges et Thuriféraire}{s'avancent solennellement devant l'autel}

\rub{tous}{se placent discrètement autour de l'autel}

\subsection{Sanctus\footnote{Par le Sanctus toute la création participe à l'action de grâce            
eucharistique. C'est une bénédiction pour magnifier l'amour trois fois
saint de Dieu.}}

\rep{Sanctus, Sanctus, D\'ominus, Deus S\'abaoth~!\\
Pleni sunt coeli et terra gl\'oria tua. \\
Hos\'anna in excelsis~!\\
Benedictus qui venit in n\'omine D\'omini. \\
Hos\'anna in excelsis~! \\ }

\rep{Saint, Saint, Saint le Seigneur, Dieu de l'univers~!\\
Le ciel et la terre sont remplis de ta gloire.\\
Hosanna au plus haut des cieux~!\\
Béni soit celui qui vient au nom du Seigneur.\\
Hosanna au plus haut des cieux~!  }

\rub{tous}{s'agenouillent en signe d'adoration}

\cel{Le prêtre dit, les mains étendues~:}

Toi qui es vraiment saint,
toi qui es la source de toute sainteté,
Seigneur, nous te prions:

\cel{Il rapproche les mains et en les tenant étendues sur les offrandes, il dit~:}

Sanctifie ces offrandes
en répandant sur elles ton Esprit; \cel{Il joint les mains}
qu'elles deviennent pour nous \cel{Il fait un signe de croix sur le pain et le calice. Il joint les mains.}
le corps et le sang de Jésus, le Christ, notre Seigneur.

\subsection{Consécration}

Toi qui es vraiment saint,
toi qui es la source de toute sainteté,
nous voici rassemblés devant toi,
et, dans la communion de toute l'Église,
en ce premier jour de la semaine
nous célébrons le jour
où le Christ est ressuscité d'entre les morts.
Par lui que tu as élevé à ta droite,
Dieu notre Père, nous te prions:

Au moment d'être livré
et d'entrer librement dans sa passion,
il prit le pain, il rendit grâce, il le rompit Il prend le pain.
et le donna à ses disciples, en disant~:
\textsc{«~Prenez, et mangez-en tous:} \cel{Il s'incline un peu.}
\textsc{ceci est mon corps livré pour vous.~\fg}

\cel{Il montre au peuple l'hostie consacrée, la repose sur la patène et fait la génuflexion. Ensuite il continue:}

De même, à la fin du repas,
il prit la coupe; \cel{Il prend le calice.}
de nouveau il rendit grâce,
et la donna à ses disciples, en disant~:
\textsc{\og~Prenez, et buvez-en tous,} \cel{Il s'incline un peu.}
\textsc{car ceci est la coupe de mon sang,
le sang de l'Alliance nouvelle et éternelle,
qui sera versé
pour vous et pour la multitude
en rémission des péchés.
Vous ferez cela,
en mémoire de moi.~\fg}

\cel{Il montre le calice au peuple, le dépose sur le corporal et fait la génuflexion.}

\subsection*{Anamnèse\expl{L'assemblée chante le mystère pascal dont la messe est le mémorial.}}

\cel{Puis le prêtre introduit une des trois acclamations suivantes, et le peuple poursuit.}

\begin{itemize}
\item  Il est grand, le mystère de la foi:\\
\rep{Nous proclamons ta mort, Seigneur Jésus,\\
nous célébrons ta résurrection,\\
nous attendons ta venue dans la gloire.}

\item Quand nous mangeons ce pain
et buvons à cette coupe,
nous célébrons le mystère de la foi:\\
\rep{Nous rappelons ta mort,\\
Seigneur ressuscité,\\
et nous attendons que tu viennes.}

\item Proclamons le mystère de la foi:\\
\rep{Gloire à toi qui étais mort,\\
gloire à toi qui es vivant,\\
notre Sauveur et notre Dieu:\\
Viens, Seigneur Jésus!}

\end{itemize}


\cel{Ensuite, les mains étendues, le prêtre dit:}

Faisant ici mémoire
de la mort et de la résurrection de ton Fils,
nous t'offrons, Seigneur,
le pain de la vie et la coupe du salut,
et nous te rendons grâce,
car tu nous as choisis pour servir en ta présence.

Humblement, nous te demandons
qu'en ayant part au corps et au sang du Christ,
nous soyons rassemblés
par l'Esprit Saint
en un seul corps.

Souviens-toi, Seigneur,
de ton Église répandue à travers le monde:
Fais-la grandir dans ta charité
avec le Pape N.,
notre évêque N.,
et tous ceux qui ont la charge de ton peuple.

Souviens-toi aussi de nos frères
qui se sont endormis
dans l'espérance de la résurrection,
et de tous les hommes qui ont quitté cette vie:
reçois-les dans ta lumière, auprès de toi.

Sur nous tous enfin,
nous implorons ta bonté:
Permets qu'avec la Vierge Marie,
la bienheureuse Mère de Dieu,
avec les Apôtres et les saints de tous les temps
qui ont vécu dans ton amitié,
nous ayons part à la vie éternelle,
et que nous chantions ta louange,
par Jésus Christ, ton Fils bien-aimé. Il joint les mains.

\subsection{Doxologie}

\emph{Doxologie veut dire parole de gloire. A la fin de la prière               
eucharistique, au nom de l'assemblée, le prêtre va adresser au Père
une parole de gloire.}

\cel{Il prend la patène avec l'hostie, ainsi que le calice, et, les élevant ensemble, il dit:}

Par lui, avec lui et en lui, à toi, Dieu le Père tout-puissant, dans
l'unité du Saint-Esprit, tout honneur et toute gloire, pour les
siècles des siècles.

{\bf Amen.}
