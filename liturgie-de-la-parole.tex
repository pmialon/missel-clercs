\section{Liturgie de la parole\footnote{L'assemblée se réunit autour des livres qui constituent la                
mémoire vivante de la Parole de Dieu.}}

\rub{cérémoniaire}{va se placer pour attendre le lecteur}

\comment{On s'assied.}

\subsection{Première lecture}

\rub{cérémoniaire}{invite le lecteur à saluer profondément l'autel puis l'accompagne à l'ambon.}

... Propre à chaque dimanche.

Parole du Seigneur.

\rep{Nous rendons grâce à Dieu.}

\subsection{Psaume}
\fixme{quid cérémoniaire?}

...

\subsection{Deuxième lecture}

... Propre à chaque dimanche.

Parole du Seigneur.

\rep{Nous rendons grâce à Dieu.}

\rub{Tous}{debout}

\rub{Thuriféraire et navette}{se présente au célébrant pour remplir et bénir l'encensoir}

\rub{premiers cierges}{se dirigent devant l'autel}

\rub{Thuriféraire}{rejoint les premiers cierges}

\rub{premiers cierges et thuriféraire}{saluent ensemble l'autel et se rendent devant l'ambon}

\subsection{Alléluia}

\comment{Durant le carême on remplace l'Alléluia par un hymne, un verset ou un trait.}

\subsection{Évangile}

\rub{Thuriféraire}{FIXME}
\fixme{ici préciser les déplacements du thuriféraire}

... Lecture propre à chaque dimanche

Acclamons la parole de Dieu.

\rep{Louange à toi, Seigneur Jésus !}

\emph{On s'assied.}

\subsection{Homélie.}

\comment{Après l'homélie on garde un moment le silence pour méditer ce que l'on vient d'entendre.}

\subsection{Credo}

\subsubsection{Symbole des Apôtres\footnote{Le Symbole de Apôtres d’une concision bien romaine pourrait remonter au deuxième siècle. }}

\rep{Credo in Deum, Patrem omnipotentem, Creatorem caeli et terrae.\\
Et in Iesum Christum, Filium eius unicum, Dominum nostrum~:\\
qui conceptus est de Spiritu Sancto, natus ex Maria Virgine,\\
passus sub Pontio Pilato, crucifixus, mortuus, et sepultus,\\
descendit ad inferos,\\
tertia die resurrexit a mortuis,\\
ascendit ad caelos,\\
sedet ad dexteram Dei Patris omnipotentis,\\
inde venturus est iudicare vivos et mortuos.\\
Credo in Spiritum Sanctum,\\
sanctam Ecclesiam catholicam,\\
sanctorum communionem,\\
remissionem peccatorum,\\
carnis resurrectionem,\\
vitam aeternam.\\
Amen.
}

\rep{
Je crois en Dieu, le Père tout-puissant, créateur du ciel et de la terre, \\
et en Jésus-Christ, son Fils unique, notre Seigneur, \\
qui a été conçu du Saint-Esprit, (et) qui est né de la Vierge Marie ; \\
(il) a souffert sous Ponce Pilate, (il) a été crucifié, (il) est mort, (il) a été enseveli, (il) est descendu aux enfers ;\\
le troisième jour, (il) est ressuscité des morts~;\\
(il) est monté au ciel, (il) est assis à la droite de Dieu, le Père tout-puissant~; \\
d'où il viendra pour juger les vivants et les morts. \\
Je crois en l'Esprit-Saint \\
à la sainte Église catholique, \\
(à) la communion des saints, \\
(à) la rémission des péchés, \\
(à) la résurrection de la chair \\
et (à) la vie éternelle. \\
Amen.
}




\subsubsection{Symbole de Nicée\footnote{Le Symbole de Nicée-Constantinople, 
plus long et plus oriental, est l’oeuvre, comme son nom l’indique, des Conciles 
de Nicée (325) et de Constantinople (381)}}

\rep{Credo in unum Deum, Patrem omnipoténtem, fact\'orem cæli et terræ, visibílium \'omnium, et invisibílium.\\
Et in unum D\'ominum Iesum Christum, Fílium Dei unigénitum.\\
Et ex Patre natum ante \'omnia sæcula. Deum de Deo, lumen de lúmine, Deum verum de Deo vero.\\
Génitum, non factum, consubstantiálem Patri~: per quem \'omnia facta sunt.\\
Qui propter nos h\'omines, et propter nostram salútem descéndit de cælis.\\
Et incarnátus est de Spíritu Sancto ex María Vírgine~: et homo factus est.\\
Crucifíxus étiam pro nobis~: sub P\'ontio Piláto passus, et sepúltus est.\\
Et resurréxit tértia die, secúndum Scriptúras.\\
Et ascéndit in cælum~: sedet ad déxteram Patris.\\
Et íterum ventúrus est cum gl\'oria iudicáre vivos, et m\'ortuos~: cuius regni non erit finis.\\
Et in Spíritum Sanctum, D\'ominum, et vivificántem~: qui ex Patre, Fili\'oque procédit.\\
Qui cum Patre, et Filio simul adorátur, et conglorificátur~: qui locútus est per Prophétas.\\
Et unam, sanctam, cath\'olicam et apost\'olicam Ecclésiam.\\
Confíteor unum baptísma in remissi\'onem peccat\'orum.\\
Et expécto resurrecti\'onem mortu\'orum. Et vitam ventúri s\'\ae{}culi.\\
Amen.
}

\rep{
Je crois en un seul Dieu, le Père tout-puissant, créateur du ciel et de la terre, de l'univers visible et invisible.\\
Je crois en un seul Seigneur, Jésus-Christ, le Fils unique de Dieu, né du Père avant tous les siècles~;\\
il est Dieu, né de Dieu, lumière, née de la lumière, vrai Dieu, né du vrai Dieu.\\
Engendré, non pas créé, de même nature que le Père, et par lui tout a été fait.\\
Pour nous les hommes, et pour notre salut, il descendit du ciel ; par l'Esprit-Saint, il a pris chair de la Vierge Marie, et s'est fait homme. \\
Crucifié pour nous sous Ponce Pilate, il souffrit sa passion et fut mis au tombeau.\\
Il ressuscita le troisième jour, conformément aux Écritures, et il monta au ciel ; il est assis à la droite du Père.\\
Il reviendra dans la gloire, pour juger les vivants et les morts~; et son règne n'aura pas de fin.\\
Je crois en l'Esprit Saint, qui est Seigneur et qui donne la vie~; il procède du Père et du Fils. \\
Avec le Père et le Fils, il reçoit même adoration et même gloire~; il a parlé par les prophètes. \\
Je crois en l'Église, une, sainte, catholique et apostolique. \\
Je reconnais un seul baptême pour le pardon des péchés. \\
J'attends la résurrection des morts, et la vie du monde à venir. \\
Amen.
}



\subsection{Prière universelle}

\rub{procession des offrandes}{se mettent en rang derrière l'autel et se rendent au lieu de départ de la procession des offrandes}

\comment{Dans la prière universelle, ou prière des fidèles, le peuple répond en quelque sorte à la parole de Dieu reçue dans la foi et, exerçant la fonction de son sacerdoce baptismal, présente à Dieu des prières pour le salut de tous. Il convient que cette prière ait lieu habituellement aux messes avec peuple, si bien que l'on fasse des supplications pour la sainte Église, pour ceux qui nous gouvernent, pour ceux qui sont accablés par diverses misères, pour tous les hommes et pour le salut du monde entier.}

\comment{Les intentions seront habituellement:\begin{enumerate}
\item pour les besoins de l'Église,
\item pour les dirigeants des affaires publiques et le salut du monde entier,
\item pour ceux qui sont accablés par toutes sortes de difficultés,
\item pour la communauté locale.
\end{enumerate}
}

... Propre à chaque messe.

\cel{Le prêtre conclut cette prière, et tous répondent~:}

\rep{Amen.}



